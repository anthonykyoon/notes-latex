%%%%%%%%%%%%%%%%%%%%%%%%%%%%%%%%%%%%%%%%%%%%%%%%%%%%%%%%%%%%%%%%%%%%%%%%%%%%%%%%%%%%
%Do not alter this block of commands.  If you're proficient at LaTeX, you may include additional packages, create macros, etc. immediately below this block of commands, but make sure to NOT alter the header, margin, and comment settings here. 
\documentclass[12pt]{article}
 \usepackage[margin=1in]{geometry} 
\usepackage{amsmath,amsthm,amssymb,amsfonts, enumitem, fancyhdr, color, comment, graphicx, environ}
\usepackage[colorinlistoftodos]{todonotes}
\pagestyle{fancy}
\setlength{\headheight}{65pt}
\newenvironment{problem}[2][Problem]{\begin{trivlist}
\item[\hskip \labelsep {\bfseries #1}\hskip \labelsep {\bfseries #2.}]}{\end{trivlist}}
\newenvironment{sol}
    {\emph{Solution:}
    }
    {
    \qed
    }
\specialcomment{com}{ \color{blue} \textbf{Comment:} }{\color{black}} %for instructor comments while grading
\NewEnviron{probscore}{\marginpar{ \color{blue} \tiny Problem Score: \BODY \color{black} }}

\newcounter{subproblem}
% \renewcommand{\thesubproblem}{\alph{subproblem}} % letters 
\renewcommand{\thesubproblem}{\arabic{subproblem}} % numbers
\newenvironment{subprob}[1][]{
  \refstepcounter{subproblem}
  \begin{trivlist}
  \item[\hskip \labelsep {\bfseries (\thesubproblem)}]
}{
  \end{trivlist}
}
\newenvironment{subsol}
    {\emph{Solution:}
    }
    {
    \qed
    }

\setlength {\marginparwidth }{2cm}
%%%%%%%%%%%%%%%%%%%%%%%%%%%%%%%%%%%%%%%%%%%%%%%%%%%%%%%%%%%%%%%%%%%%%%%%%%%%%%%%%

\newcommand{\R}{\mathbb{R}}
\newcommand{\N}{\mathbb{N}}
\newcommand{\Q}{\mathbb{Q}}
\newcommand{\Z}{\mathbff{Z}}
\newcommand{\st}{\text{s.t}}



%%%%%%%%%%%%%%%%%%%%%%%%%%%%%%%%%%%%%%%%%%%%%
%Fill in the appropriate information below
\lhead{Class: ECON 20210}
\chead{Assignment: 1}
% \lhead{Anthony Yoon}  %replace with your name
\rhead{Anthony Yoon} %replace XYZ with the homework course number, semester (e.g. ``Spring 2019"), and assignment number.
%%%%%%%%%%%%%%%%%%%%%%%%%%%%%%%%%%%%%%%%%%%%%


%%%%%%%%%%%%%%%%%%%%%%%%%%%%%%%%%%%%%%
%Do not alter this block.
\begin{document}
%%%%%%%%%%%%%%%%%%%%%%%%%%%%%%%%%%%%%%


%Solutions to problems go below.  Please follow the guidelines from https://www.overleaf.com/read/sfbcjxcgsnsk/


%Copy the following block of text for each problem in the assignment.
\begin{problem}{1} 
MATLAB Tutorial
\end{problem}
\begin{sol}
N/A
\end{sol}

\setcounter{subproblem}{0}
\begin{problem}{2} 
Dynamics in a Non-linear system. 
\end{problem}
\begin{subprob}
123
\end{subprob}
\begin{subsol}

\end{subsol}

\begin{problem}{3} 
Level vs. Growth Rate
\end{problem}
\begin{sol}
Write your solution here. 
\end{sol}

\setcounter{subproblem}{0}
\begin{problem}{4} 
Real GDP as a measure of welfare?
\end{problem}
\begin{subprob}
\end{subprob}
\begin{subsol}
Setting up the optimization problem:
\begin{align*}
    \max & \quad U\\
    \st & \quad \sum_{i = 1}^{n} p_i x_i = M
\end{align*}
with the following FOCs 
\[
[x_i] \quad \frac{\partial U}{\partial x_i} = \lambda p_i
\]
and $\lambda$ is the marginal utility of income.
\end{subsol}
\begin{subprob}
\end{subprob}
\begin{subsol}
    Taking the total differenial of $U$, we can see that:
    \[
    dU(x_1, x_2, \dots, x_n) = \sum_{i}^{n} \frac{\partial U}{\partial x_i} dx_i
    \]
\end{subsol}
\begin{subprob}
\end{subprob}
\begin{subsol}
    Since $\frac{\partial U}{\partial x_i} = \lambda p_i$ at the optimum, we see that:
    \[
        dU = \sum_{i}^{n} \frac{\partial U}{\partial x_i} dx_i = \lambda \sum_{i}^{n} p_i dx_i
    \]
    The statements holds to be true.
\end{subsol}

\setcounter{subproblem}{0}

\begin{problem}{5} 
Intertemporal Consumption Choice
\end{problem}
\begin{subprob}
\end{subprob}
\begin{subsol}
    \begin{align*}
         A_0 K^\alpha_0 &= c_0 + K_1\\
         A_1 K_1^\alpha &= c_1 
    \end{align*}
    as in time period 1, the individual consumes all of $y$
\end{subsol}
\begin{subprob}
    \[
    y_0 = c_0 + K_1 \iff y_0 - c_0 = K_1
    \]
    thus,
    \[
    A_1(y_0 - c_0)^\alpha = c_1
    \]
\end{subprob}
\begin{subprob}
    Draw the graph \todo{Draw the graph}
\end{subprob}
\begin{subsol}
    See graph. The slope represents the rate of change between the consumption in the current time period and the next time period. 
\end{subsol}
\begin{subprob}
\end{subprob}
\begin{subsol}
    See graph
\end{subsol}
\begin{subprob}
    \todo{draw the graph.}
\end{subprob}
\begin{subsol}
    See graph above. If $A_0$ increases, $y_0$ increases as technology in the current period would increase. This implies that budget constraint shifts outward to the right. Since the consumer now has more "budget" of corn, he now was more income to consume more. Thus, 
    \begin{itemize}
        \item $y_0$ increases
        \item $c_0$ increases
        \item $c_1$ increases
    \end{itemize}
\end{subsol}
\begin{subprob}
    \todo{Draw Graph}
\end{subprob}
\begin{subsol}
    See graph above. If $A_1$ increases, this means that $y_1$ strictly increases. This implies that the maximum possible value of $c_1$ will increases, and make the graph steeper. Thus, 
    \begin{itemize}
        \item $c_1$ increases
        \item $y_1$ increases
        \item $c_0$ does not change, as production is not impacted. 
    \end{itemize}
\end{subsol}
\begin{subprob}
    Algebra manipulation. 
\end{subprob}
\begin{subsol}
    We are interested in the following optimization problem:
    \begin{align*}
        \max & \quad \ln(c_0) + \beta \ln(c_1)\\
        \st & \quad  A_1(y_0 - c_0)^\alpha = c_1
    \end{align*}
    Subsituting the constraint into the objective function yields:
    \[
    \ln (c_0) + \beta \ln( A_1(y_0 - c_0)^\alpha)
    \]
    taking the deriative with respect to $c_0$ yields:
    \[
    \frac{1}{c_0} - \frac{\alpha \beta}{c_0 - y_0} = 0
    \]
    which, after some algebra, yields:
    \[
    c_0^* = \frac{y_0}{1 + \beta \alpha}
    \]
    Thus, this implies that 
    \[
    c_1^* = A_1 \left( \frac{\alpha \beta y_0}{1 + \beta \alpha} \right)^\alpha
    \]
    Thus, we can see that since $A_1$ only appears in $c_1$, this implies that $c_1$ and $y_1$ increases. Additionally, we can see that increasing $A_0$ would increase $y_0$ which in turn increase all values. 
\end{subsol}


\setcounter{subproblem}{0}
\begin{problem}{6} 
Exact Price Index from the Economic Approach
\end{problem}
\begin{subprob}
\end{subprob}
\begin{subsol}
    \begin{align*}
        \max & \quad \ln x + \ln y\\
        \st & \quad p_x x + p_y y = M
    \end{align*}
    we have the following FOCs:
    \begin{align*}
        [x] & \quad \frac{1}{x} = \lambda p_x\\
        [y] & \quad \frac{1}{y} = \lambda p_y \\ 
        [\lambda] & \quad M = p_x x + p_y y
    \end{align*}
    Note that the FOCs imply that $p_x x = p_y y$ and thus, using the budget constraint, we find that:
    \[
    p_x x + p_y y = M \iff 2p_x x = M \iff x = \frac{M}{2p_x}
    \]
    and by symmetry
    \[
    y = \frac{M}{2p_y}
    \]
    which indicates that expenditure share is one half. 
\end{subsol}
\begin{subprob}
\end{subprob}
\begin{subsol}
    The indirect utilty function is 
    \[
    V(M, P) = \ln \left( \frac{M}{2p_x} \right) + \ln \left( \frac{M}{2p_y} \right)
    \]
\end{subsol}
\begin{subprob}
\end{subprob}
\begin{subsol}
    We aim to use duality to prove this. Let 
    \[
    U = 2\ln(M) - \ln(4) -\ln(p_x p_y)
    \]
\end{subsol}
\begin{subprob}
\end{subprob}
\begin{subsol}
    Using duality, we see that:
    \begin{align*}
        U &= 2\ln(M) - \ln(4) - \ln (p_x p_y)\\
        \ln(M) &= 0.5 \ln(4) + \ln ( \sqrt{p_x p_y})\\
        M &= 2e^\frac{U}{2} \sqrt{p_x p_y}
    \end{align*}
    Using Shephard's Lemma, we see that:
    \[
    x^h = \frac{\partial e(p, U)}{\partial p_x} = \sqrt{\frac{e^U p_y}{p_x}}
    \]
    and by symmetry
    \[
    y^h = \frac{\partial e(p,U)}{\partial p_y} = \sqrt{\frac{e^U p_x}{p_y}}
    \]
\end{subsol}
\begin{subprob}
\end{subprob}
\begin{subsol}
    \[
    M = \frac{2e^\frac{U_t}{2} \sqrt{p_x^t p_y^t}}{2e^\frac{U_0}{2} \sqrt{p_x^0 p_y^0}} = e^\frac{U_t - U_0}{2} \sqrt{\frac{p^t_x p^t_y}{p_x^0p_y^0}}
    \]\todo{Check this}
\end{subsol}
\begin{subprob}
\end{subprob}
\begin{subsol}
    \todo{Double check this} This is the fisher price index. 
\end{subsol}
%%%%%%%%%%%%%%%%%%%%%%%%%%%%%%%%%%%%%
%Do not alter anything below this line.
\end{document}