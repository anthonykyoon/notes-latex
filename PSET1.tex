%%%%%%%%%%%%%%%%%%%%%%%%%%%%%%%%%%%%%%%%%%%%%%%%%%%%%%%%%%%%%%%%%%%%%%%%%%%%%%%%%%%%
%Do not alter this block of commands.  If you're proficient at LaTeX, you may include additional packages, create macros, etc. immediately below this block of commands, but make sure to NOT alter the header, margin, and comment settings here. 
\documentclass[12pt]{article}
 \usepackage[margin=1in]{geometry} 
\usepackage{amsmath,amsthm,amssymb,amsfonts, enumitem, fancyhdr, color, comment, graphicx, environ, float}
\usepackage[colorinlistoftodos]{todonotes}
\pagestyle{fancy}
\setlength{\headheight}{65pt}
\newenvironment{problem}[2][Problem]{\begin{trivlist}
\item[\hskip \labelsep {\bfseries #1}\hskip \labelsep {\bfseries #2.}]}{\end{trivlist}}
\newenvironment{sol}
    {\emph{Solution:}
    }
    {
    \qed
    }
\specialcomment{com}{ \color{blue} \textbf{Comment:} }{\color{black}} %for instructor comments while grading
\NewEnviron{probscore}{\marginpar{ \color{blue} \tiny Problem Score: \BODY \color{black} }}

\newcounter{subproblem}
% \renewcommand{\thesubproblem}{\alph{subproblem}} % letters 
\renewcommand{\thesubproblem}{\arabic{subproblem}} % numbers
\newenvironment{subprob}[1][]{
  \refstepcounter{subproblem}
  \begin{trivlist}
  \item[\hskip \labelsep {\bfseries (\thesubproblem)}]
}{
  \end{trivlist}
}
\newenvironment{subsol}
    {\emph{Solution:}
    }
    {
    \qed
    }

\setlength {\marginparwidth }{2cm}
%%%%%%%%%%%%%%%%%%%%%%%%%%%%%%%%%%%%%%%%%%%%%%%%%%%%%%%%%%%%%%%%%%%%%%%%%%%%%%%%%

\newcommand{\R}{\mathbb{R}}
\newcommand{\N}{\mathbb{N}}
\newcommand{\Q}{\mathbb{Q}}
\newcommand{\Z}{\mathbff{Z}}
\newcommand{\st}{\text{s.t}}


\usepackage{listings}
\usepackage{xcolor}
\lstset{
  basicstyle=\ttfamily\small,
  backgroundcolor=\color{gray!10},
  frame=single,
  breaklines=true,
  keywordstyle=\color{blue},
  commentstyle=\color{gray},
  stringstyle=\color{red},
  showstringspaces=false,
  language=Matlab
}

%%%%%%%%%%%%%%%%%%%%%%%%%%%%%%%%%%%%%%%%%%%%%
%Fill in the appropriate information below
\lhead{Class: ECON 20210}
\chead{Assignment: 1}
% \lhead{Anthony Yoon}  %replace with your name
\rhead{Anthony Yoon \\ Min Seo Kim \\ Sam Konkel \\ Pratyush Sharma} %replace XYZ with the homework course number, semester (e.g. ``Spring 2019"), and assignment number.
%%%%%%%%%%%%%%%%%%%%%%%%%%%%%%%%%%%%%%%%%%%%%


%%%%%%%%%%%%%%%%%%%%%%%%%%%%%%%%%%%%%%
%Do not alter this block.
\begin{document}
%%%%%%%%%%%%%%%%%%%%%%%%%%%%%%%%%%%%%%


%Solutions to problems go below.  Please follow the guidelines from https://www.overleaf.com/read/sfbcjxcgsnsk/


%Copy the following block of text for each problem in the assignment.
% \begin{problem}{1} 
% MATLAB Tutorial
% \end{problem}
% \begin{sol}
% N/A
% \end{sol}

\setcounter{subproblem}{0}
\begin{problem}{2} 
Dynamics in a Non-linear system. 
\end{problem}
\begin{subprob}
\end{subprob}
\begin{subsol}
    For the graph to have a stable solution, the graph must asymptoically approach a value, to encourage convergance. 
\end{subsol}
\begin{subprob}
\end{subprob}
\begin{subsol}
We can consider cases: $|a| < 1, |a| = 1, |a| > 1$. Note that if $|a| = 1$, we have the linear difference equation $x_{t_1} = x_t + b$, which implies that there is not stable solution, and thus approachs infinity. If $|a| < 1$ or $|a| > 1$, this means that the exponetial term grows slower and faster than the linear term respectively. However, as noted in \textbf{1}, we can see that if the slope of graph $|ax_t^{a-1}| < 1$, we will asymptocally approach a solution. Else, we may diverge and not
approach a steady state solution. Regrading the trajectories, we see that there are types of convergance. Either the function approachs the stable state monotonically or the function oscillates towards the final stable state solution.  
\end{subsol}
\begin{subprob}
\end{subprob}
\begin{subsol}
    \begin{align*}
        \overline{x} &= \overline{x}_t^{0.5} + 2\\
        \overline{x} - \overline{x}_t^{0.5} &= 2\\
        (\overline{x}^{0.5} - 2)(\overline{x}^{0.5} + 1) &= 0\\ 
    \end{align*}
    The only possible solution to the above is $\overline{x} = 4$
\end{subsol}

\begin{subprob}
\end{subprob}
\begin{subsol}
    \begin{figure}[H]
        \centering
        \includegraphics[width=0.75\linewidth]{trajectory_plot.pdf}
        \caption{Intial value of 0.1}
        \label{fig:1}
    \end{figure}
    \newpage
    Code for the above simulation: 
    \begin{lstlisting}
        function predicted = predict_return(past_returns)
        a = 0.5;
        b = 2; 
        N = 100; 
        Y = zeros(N+1, 1);
        y_0 = 0.1;
        % Create the X-axis
        X = 0:1:N;
        X = X';
        % Calculate the dynamics and record it in the Y vector
        Y(1) = y_0;
        for i = 1:N
        Y(i+1) = Y(i)^a +b;
        end
        Y;
        plot(X,Y)
        xlabel('Time (t)')
        ylabel('x_t')
        title('Dynamics of x_{t+1} = x_t^{0.5} + 2 with intial value 0.1')
        print('trajectory_plot3','-dpdf')
        \end{lstlisting}
    \begin{figure}[H]
        \centering
        \includegraphics[width=0.75\linewidth]{trajectory_plot1.pdf}
        \caption{Intial Value of 10}
        \label{fig:12121}
    \end{figure}
    \newpage
    Code for the above simulation:
    \begin{lstlisting}
        function predicted = predict_return(past_returns)
        a = 0.5;
        b = 2; 
        N = 100; 
        Y = zeros(N+1, 1);
        y_0 = 10;
        % Create the X-axis
        X = 0:1:N;
        X = X';
        % Calculate the dynamics and record it in the Y vector
        Y(1) = y_0;
        for i = 1:N
        Y(i+1) = Y(i)^a +b;
        end
        Y;
        plot(X,Y)
        xlabel('Time (t)')
        ylabel('x_t')
        title('Dynamics of x_{t+1} = x_t^{0.5} + 2 with intial value 10')
        print('trajectory_plot3','-dpdf')
    \end{lstlisting}
\end{subsol}
\begin{subprob}
\end{subprob}
\begin{subsol}
    \begin{figure}[H]
        \centering
        \includegraphics[width=0.75\linewidth]{trajectory_plot2.pdf}
        \caption{Intial value of 0.1}
        \label{fig:3}
    \end{figure}
    \newpage 
    Code for then above simulation:
    \begin{lstlisting}
        function predicted = predict_return(past_returns)
        a = -0.9;
        b = 2; 
        N = 100; 
        Y = zeros(N+1, 1);
        y_0 = 0.1;
        % Create the X-axis
        X = 0:1:N;
        X = X';
        % Calculate the dynamics and record it in the Y vector
        Y(1) = y_0;
        for i = 1:N
        Y(i+1) = Y(i)^a +b;
        end
        Y;
        plot(X,Y)
        xlabel('Time (t)')
        ylabel('x_t')
        title('Dynamics of x_{t+1} = x_t^{-0.9} + 2 with intial value 0.1')
        print('trajectory_plot3','-dpdf')
    \end{lstlisting}
    \begin{figure}[H]
        \centering
        \includegraphics[width=0.75\linewidth]{trajectory_plot3.pdf}
        \caption{Intial Value of 10}
        \label{fig:enter-label}
    \end{figure}
    \newpage
    \begin{lstlisting}
        function predicted = predict_return(past_returns)
        a = 0.5;
        b = 2; 
        N = 100; 
        Y = zeros(N+1, 1);
        y_0 = 10;
        % Create the X-axis
        X = 0:1:N;
        X = X';
        % Calculate the dynamics and record it in the Y vector
        Y(1) = y_0;
        for i = 1:N
        Y(i+1) = Y(i)^a +b;
        end
        Y;
        plot(X,Y)
        xlabel('Time (t)')
        ylabel('x_t')
        title('Dynamics of x_{t+1} = x_t^{-0.9} + 2 with intial value 10')
        print('trajectory_plot3','-dpdf')
    \end{lstlisting}
    The steady state changes to approximately 2.45. Both trajectories seem to oscillate towards to the steady state, which is a change from the previous set of paraneters, which monotonically convereg towards to the stable solution. 
\end{subsol}
\begin{subprob}
    
\end{subprob}
\begin{subsol}
    We are given $x_{t+1} = x_t^a + b$ where $f(x) = x^a + b$. Let $\overline{x}$ such that $f(\overline{x}) = \overline{x}$ and $\overline{x} \in (x_t - \epsilon, x_t + \epsilon), \forall \epsilon >0 $. Consider the first degree Taylor expansion around $\overline{x}$. We can see that:
    \begin{align*}
        x_{t+1} &\approx f(\overline{x}) + f'(x)(x_t-\overline{x})\\
        x_{t+1} - \overline{x} &\approx f'(x)(x_t - \overline{x})
    \end{align*}
    Let $d_{t} = x_{t} - \overline{x}$ and similarly for $d_{t+1}$. Thus, 
    \[
        x_{t+1} - \overline{x} \approx f'(x)(x_t - \overline{x}) \implies d_{t+1} = f'(x)d_t
    \]
    which implies that the general solution:
    \[
    d_t = (f'(\overline{x}))^t d_0
    \]
    Using this equation, we see that:
    \begin{align*}
        d_t &= (f'(\overline{x}))^t d_0\\ 
        (x_t - \overline{x}) &= (f'(\overline{x}))^t(x_0 - \overline{x})\\
        x_t &= (f'(\overline{x}))^t x_0 + \overline{x}(1-f'(x)^t)
    \end{align*}
    Note that $f'(x) = -0.5x_t^{-0.5}$ and for any $x \in [0, \infty]$ that $|f'(x)| < 1$. Thus, for any $t$ sufficently large enough, we see that $x_t \to \overline{x}$, which indicates that this equation approximates it very well. 
\end{subsol}

% \begin{problem}{3} 
% Level vs. Growth Rate
% \end{problem}
% \begin{sol}
% Not to be submitted
% \end{sol}

\setcounter{subproblem}{0}
\begin{problem}{4} 
Real GDP as a measure of welfare?
\end{problem}
\begin{subprob}
\end{subprob}
\begin{subsol}
Setting up the optimization problem:
\begin{align*}
    \max & \quad U\\
    \st & \quad \sum_{i = 1}^{n} p_i x_i = M
\end{align*}
with the following FOCs 
\[
[x_i] \quad \frac{\partial U}{\partial x_i} = \lambda p_i
\]
and $\lambda$ is the marginal utility of income.
\end{subsol}
\begin{subprob}
\end{subprob}
\begin{subsol}
    Taking the total differenial of $U$, we can see that:
    \[
    dU(x_1, x_2, \dots, x_n) = \sum_{i}^{n} \frac{\partial U}{\partial x_i} dx_i
    \]
\end{subsol}
\begin{subprob}
\end{subprob}
\begin{subsol}
    Since $\frac{\partial U}{\partial x_i} = \lambda p_i$ at the optimum, we see that:
    \[
        dU = \sum_{i}^{n} \frac{\partial U}{\partial x_i} dx_i = \lambda \sum_{i}^{n} p_i dx_i
    \]
    The statements holds to be true.
\end{subsol}

\setcounter{subproblem}{0}

\begin{problem}{5} 
Intertemporal Consumption Choice
\end{problem}
\begin{subprob}
\end{subprob}
\begin{subsol}
    \begin{align*}
         A_0 K^\alpha_0 &= c_0 + K_1\\
         A_1 K_1^\alpha &= c_1 
    \end{align*}
    as in time period 1, the individual consumes all of $y$
\end{subsol}
\begin{subprob}
    \[
    y_0 = c_0 + K_1 \iff y_0 - c_0 = K_1
    \]
    thus,
    \[
    A_1(y_0 - c_0)^\alpha = c_1
    \]
\end{subprob}
\begin{figure}[H]
    \centering
    \includegraphics[width=0.75\linewidth]{IMG_8FBD37C0B675-1.jpeg}
    \caption{Graph for 3,4,5,6}
    \label{fig:enter-label1}
\end{figure}
\begin{subprob}
\end{subprob}
\begin{subsol}
    See graph above in black ink. The slope represents the rate of change between the consumption in the current time period and the next time period. 
\end{subsol}
\begin{subprob}
\end{subprob}
\begin{subsol}
    See graph above in black ink
\end{subsol}
\begin{subprob}
\end{subprob}
\begin{subsol}
    See graph above, blue line. If $A_0$ increases, $y_0$ increases as technology in the current period would increase. This implies that budget constraint shifts outward to the right. Since the consumer now has more "budget" of corn, he now was more income to consume more. Thus, 
    \begin{itemize}
        \item $y_0$ increases
        \item $c_0$ increases
        \item $c_1$ increases
    \end{itemize}
\end{subsol}
\begin{subprob}
\end{subprob}
\begin{subsol}
    See graph above, red line. If $A_1$ increases, this means that $y_1$ strictly increases. This implies that the maximum possible value of $c_1$ will increases, and make the graph steeper. Thus, 
    \begin{itemize}
        \item $c_1$ increases
        \item $y_1$ increases
        \item $c_0$ decreases, as consumer will substitute away from the good. 
    \end{itemize}
\end{subsol}
\begin{subprob}
    Algebra manipulation. 
\end{subprob}
\begin{subsol}
    We are interested in the following optimization problem:
    \begin{align*}
        \max & \quad \ln(c_0) + \beta \ln(c_1)\\
        \st & \quad  A_1(y_0 - c_0)^\alpha = c_1
    \end{align*}
    Subsituting the constraint into the objective function yields:
    \[
    \ln (c_0) + \beta \ln( A_1(y_0 - c_0)^\alpha)
    \]
    taking the deriative with respect to $c_0$ yields:
    \[
    \frac{1}{c_0} - \frac{\alpha \beta}{c_0 - y_0} = 0
    \]
    which, after some algebra, yields:
    \[
    c_0^* = \frac{y_0}{1 + \beta \alpha}
    \]
    Thus, this implies that 
    \[
    c_1^* = A_1 \left( \frac{\alpha \beta y_0}{1 + \beta \alpha} \right)^\alpha
    \]
    Thus, we can see that since $A_1$ only appears in $c_1$, this implies that $c_1$ and $y_1$ increases. Additionally, we can see that increasing $A_0$ would increase $y_0$ which in turn increase all values. 
\end{subsol}


\setcounter{subproblem}{0}
\begin{problem}{6} 
Exact Price Index from the Economic Approach
\end{problem}
\begin{subprob}
\end{subprob}
\begin{subsol}
    \begin{align*}
        \max & \quad \ln x + \ln y\\
        \st & \quad p_x x + p_y y = M
    \end{align*}
    we have the following FOCs:
    \begin{align*}
        [x] & \quad \frac{1}{x} = \lambda p_x\\
        [y] & \quad \frac{1}{y} = \lambda p_y \\ 
        [\lambda] & \quad M = p_x x + p_y y
    \end{align*}
    Note that the FOCs imply that $p_x x = p_y y$ and thus, using the budget constraint, we find that:
    \[
    p_x x + p_y y = M \iff 2p_x x = M \iff p_x x = \frac{M}{2} 
    \]
    and by symmetry
    \[
    p_y y = \frac{M}{2}
    \]
    which indicates that expenditure share is one half. 
\end{subsol}
\begin{subprob}
\end{subprob}
\begin{subsol}
    The indirect utilty function is 
    \[
    V(M, P) = \ln \left( \frac{M}{2p_x} \right) + \ln \left( \frac{M}{2p_y} \right)
    \]
\end{subsol}
\begin{subprob}
\end{subprob}
\begin{subsol}
    We aim to use duality to prove this. Let 
    \[
    U = 2\ln(M) - \ln(4) -\ln(p_x p_y)
    \]
\end{subsol}
\begin{subprob}
\end{subprob}
\begin{subsol}
    Using duality, we see that:
    \begin{align*}
        U &= 2\ln(M) - \ln(4) - \ln (p_x p_y)\\
        \ln(M) &= 0.5 \ln(4) + \ln ( \sqrt{p_x p_y})\\
        M &= 2e^\frac{U}{2} \sqrt{p_x p_y}
    \end{align*}
    Using Shephard's Lemma, we see that:
    \[
    x^h = \frac{\partial e(p, U)}{\partial p_x} = \sqrt{\frac{e^U p_y}{p_x}}
    \]
    and by symmetry
    \[
    y^h = \frac{\partial e(p,U)}{\partial p_y} = \sqrt{\frac{e^U p_x}{p_y}}
    \]
\end{subsol}
\begin{subprob}
\end{subprob}
\begin{subsol}
    \[
    M = \frac{2e^\frac{U_t}{2} \sqrt{p_x^t p_y^t}}{2e^\frac{U_0}{2} \sqrt{p_x^0 p_y^0}} = e^\frac{U_t - U_0}{2} \sqrt{\frac{p^t_x p^t_y}{p_x^0p_y^0}}
    \]
    Since the level of utility remains fixed, we can see that we are left with the following expression. 
    \[
        \sqrt{\frac{p^t_x p^t_y}{p_x^0p_y^0}}
    \]
\end{subsol}
\begin{subprob}
\end{subprob}
\begin{subsol}
    The price index based on a same basket given by:
    \[
    \frac{p_x^t x_0 + p_y^t y_0}{p_x^0 x_0 + p_y^0 y_0}
    \]
    However, since 
    \[
    x_0 = \frac{M}{2p_x^0} \quad y_0 = \frac{M}{2p_y^0}
    \]
    After some algebra, we see that:
    \[
    \frac{1}{2} \left( \frac{p_x^t}{p_x^0} + \frac{p_y^t}{p_y^0} \right)
    \]
    This is the geometric mean. The equation derived in \textbf{(5)} is the Fisher Price index, which is the geometric mean. The Fisher Price index takes into account the substitution effect where as the price index we just derived does not. Additionally, by the AM - GM inequality, we know that the the price index we derived will always be bigger than the fisher price index. 
\end{subsol}
%%%%%%%%%%%%%%%%%%%%%%%%%%%%%%%%%%%%%
%Do not alter anything below this line.
\end{document}
