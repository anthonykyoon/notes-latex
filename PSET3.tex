%%%%%%%%%%%%%%%%%%%%%%%%%%%%%%%%%%%%%%%%%%%%%%%%%%%%%%%%%%%%%%%%%%%%%%%%%%%%%%%%%%%%
%Do not alter this block of commands.  If you're proficient at LaTeX, you may include additional packages, create macros, etc. immediately below this block of commands, but make sure to NOT alter the header, margin, and comment settings here. 
\documentclass[12pt]{article}
 \usepackage[margin=1in]{geometry} 
\usepackage{amsmath,amsthm,amssymb,amsfonts, enumitem, fancyhdr, color, comment, graphicx, environ}
\usepackage{float}
\usepackage[colorinlistoftodos]{todonotes}

\pagestyle{fancy}
\setlength{\headheight}{65pt}
\newenvironment{problem}[2][Problem]{\begin{trivlist}
\item[\hskip \labelsep {\bfseries #1}\hskip \labelsep {\bfseries #2.}]}{\end{trivlist}}
\newenvironment{sol}
    {\emph{Solution:}
    }
    {
    \qed
    }
\specialcomment{com}{ \color{blue} \textbf{Comment:} }{\color{black}} %for instructor comments while grading
\NewEnviron{probscore}{\marginpar{ \color{blue} \tiny Problem Score: \BODY \color{black} }}

\newcounter{subproblem}
% \renewcommand{\thesubproblem}{\alph{subproblem}} % letters 
\renewcommand{\thesubproblem}{\arabic{subproblem}} % numbers
\newenvironment{subprob}[1][]{
  \refstepcounter{subproblem}
  \begin{trivlist}
  \item[\hskip \labelsep {\bfseries (\thesubproblem)}]
}{
  \end{trivlist}
}
\newenvironment{subsol}
    {\emph{Solution:}
    }
    {
    \qed
    }
%%%%%%%%%%%%%%%%%%%%%%%%%%%%%%%%%%%%%%%%%%%%%%%%%%%%%%%%%%%%%%%%%%%%%%%%%%%%%%%%%
\setlength {\marginparwidth }{2cm}


\usepackage{listings}
\usepackage{xcolor}
\lstset{
  basicstyle=\ttfamily\small,
  backgroundcolor=\color{gray!10},
  frame=single,
  breaklines=true,
  keywordstyle=\color{blue},
  commentstyle=\color{gray},
  stringstyle=\color{red},
  showstringspaces=false,
  numbers=left,            
  numberstyle=\tiny\color{gray}, 
  numbersep=10pt 
}

\newcommand{\R}{\mathbb{R}}
\newcommand{\N}{\mathbb{N}}
\newcommand{\Q}{\mathbb{Q}}
\newcommand{\Z}{\mathbff{Z}}
\newcommand{\st}{\text{s.t}}


%%%%%%%%%%%%%%%%%%%%%%%%%%%%%%%%%%%%%%%%%%%%%
%Fill in the appropriate information below
\lhead{Class: ECON 20210}
\chead{Assignment: 3}
\rhead{Anthony Yoon \\ Min Seo Kim \\ Pratuysh Sharma \\ Sam Konkel} %replace XYZ with the homework course number, semester (e.g. ``Spring 2019"), and assignment number.
%%%%%%%%%%%%%%%%%%%%%%%%%%%%%%%%%%%%%%%%%%%%%    


%%%%%%%%%%%%%%%%%%%%%%%%%%%%%%%%%%%%%%
%Do not alter this block.
\begin{document}
%%%%%%%%%%%%%%%%%%%%%%%%%%%%%%%%%%%%%%

%Copy the following block of text for each problem in the assignment.
\setcounter{subproblem}{0}
\begin{problem}{1}

\end{problem}
\begin{subprob}
    
\end{subprob}
\begin{subsol}
    The Langrangian is as follows:
    \[
    L = \sum_{t = 0}^{\infty} \beta^t u(c_t) + \sum_{i=0}^{\infty} \lambda_t (w_t - c_t - w_{t+1}) 
    \]
    with the following FoCs
    \begin{align*}
        [c_t] & \quad \beta^t u(c_t) = \lambda_t\\
        [\lambda_t] & \quad w_t = c_t + w_{t+1}\\
        [w_{t+1}] & \quad \lambda_{t+1} = \lambda_t
    \end{align*}
    We can see that we can derive the Euler equation,
    \begin{align*}
        \beta^{t+1} u'(c_t) &= \beta^t u'(c_t)\\
        \beta u'(c_{t+1}) &= u'(c_t)
    \end{align*}
\end{subsol}
\begin{subprob}
    
\end{subprob}
\begin{subsol}
    We aim to extend the findings from the Finite Horizon model to the Infinite Horizon Model. Consider the following version of the finte horizon model. 
    \[
    L = \sum_{t = 0}^{T} \left \{ \beta^T u(c_t) + \lambda_t (w_t - w_{t+1} - c_t) + \mu w_{t+1}\right\}
    \]
    where we impose the condition $w_{t+1} \geq 0$. We have the following FOCs:
    \begin{align*}
        [c_t] & \quad \beta t u'(c_t) - \lambda_t = 0\\
        [w_t] & \quad \lambda_t - \lambda_{t-1} = 0, \forall t \in \{0,1,2, \dots, T\}\\
        [w_{T+1}] & \quad -\lambda_T + \mu = 0
    \end{align*}
    with the complentary slackness condition where $\mu w_{t+1} = 0$. From the FOCs, we see that 
    \[
    \lambda_t \lambda_{t+1} \implies \lambda w_{t+1} = 0 \implies \beta^t u'(c_t) w_{t+1} = 0 
    \]
    and when we extend this idea to the infinite horizon, the following holds:
    \[
    \lim_{t \to \infty} \beta^t u'(c_t) w_{t+1} = 0
    \]
    Note that this is more of an assumption that we impose in the derivation. The intuition of this is that in the long run, we consume all our resources such that we maximize our utility in the long run.  We can see this in the following manner, assume that:
    \[
    \lim_{n \to \infty} \beta^t u'(c_t) < 0
    \]
    the above does not hold from the assumptions on the utility function. 
    \[
        \lim_{n \to \infty} \beta^t u'(c_t) > 0
    \]
    this implies that more room for consumption which implies that there can be more utility be consumed. Hence, by letting the limit equal 0, we imply there are no more room for optimization for the maximization of utility. 
\end{subsol}
\begin{subprob}
\end{subprob}
\begin{subsol}
    $u(c) = \ln (c)$ and thus $u'(c) = \frac{1}{c}$ Therefore, using the Euler Equation, we find that:
    \begin{align*}
        u'(c_t) &= \beta u'(c_{t+1})\\
        \frac{1}{c_t} &= \frac{\beta}{c_{t+1}}\\
        c_{t+1} &= \beta c_t
    \end{align*}
    Note that if the consumer wants to maximize their utility, they should aim to consume all the goods. Thus, 
    \[
    \sum_{t = 0}^{\infty} c_t = w_0
    \]
    expanding the above, we can see that: 
    \[
    \sum_{t = 0 }^{\infty} \beta^t c_0 = w_t \iff c_0 = (1-\beta)w_0 
    \] 
    this implies that 
    \[
    c_t = \beta^t (1-\beta)w_0
    \]
\end{subsol}
\newpage
\setcounter{subproblem}{0}
\begin{problem}{2}
    
\end{problem}
\begin{subprob}
    
\end{subprob}
\begin{subsol}
    Since $K_{t+1} - (1-\delta)K_t = I_t$, we find that:
    \begin{align*}
        \max & \quad \sum_{t = 0 }^{\infty} \beta^t \ln(c_t)\\
        \st & \quad C_t + K_{t+1} - (1-\delta)K_t + G_t = A_{t}^{1-\alpha}K_t^\alpha
    \end{align*}
    with the following Langrangian:
    \[
    L = \sum_{t=0}^{\infty} \beta^t \ln(c_t) + \sum_{t = 0}^{\infty} \lambda_t (A_t^{1-\alpha}K_t^{\alpha} - C_t - K_{t+1} + (1-\delta)K_t - G_t)
    \]
    Thus, we derive the following FOCs:
    \begin{align*}
        [C_t] & \quad \frac{\beta^t}{C_t} = \lambda_t\\
        [K_{t+1}] &  \quad  \lambda_{t+1} ((1-\delta) + \alpha A^{1-\alpha}_{t+1} K_{t+1}^{\alpha -1}) = \lambda_t
    \end{align*}
\end{subsol}

\begin{subprob}
    
\end{subprob}
\begin{subsol}
    Note using the $[C_t]$ and the $[K_{t+1}]$ conditions, we can derive the following:
    \[
    \frac{1}{C_t} = \frac{\beta}{C_{t+1}} (\alpha A^{1-\alpha}_{t+1}K_{t+1}^{\alpha - 1} + 1 - \delta)
    \]
    Intuively, we can see that at optimum, consuming today must have have the same marginal benefit as consuming in the next time period, assuming we adjust for all discount factors and such. 
\end{subsol}
\begin{subprob}
    
\end{subprob}
\begin{subsol}
    Define $k_t = \frac{K_t}{A_t}$, $c_t = \frac{C_t}{A_t}$, $y_t = \frac{Y_t}{A_t}$, $g_t = \frac{G_t}{A_t}$. Using the budget constraint, we can see:
    \begin{align*}
        A_t \left( \frac{K_t^\alpha}{A_t^\alpha} \right) &= C_t + K_{t+1} - K_t (1-\delta) + G_t\\
        k_t^\alpha &= c_t + K_{t+1} A_t^{-1} - k_t(1-\delta) + g_t\\
        k_t^\alpha &= c_t + \frac{K_{t+1} A_{t+1}}{A_{t+1}A_t} - k_t (1-\delta) + g_t\\
        k_t^\alpha &= c_t + k_{t+1} (1+\gamma) - k_t(1-\delta) + g_t\\
        k_t^\alpha &= c_t + k_{t+1}(\gamma + \delta) + g_t
    \end{align*}
    Using the Euler's equation, we find that 
    \begin{align*}
        \frac{1}{C_t} &= \frac{\beta}{C_{t+1}} (\alpha A^{1-\alpha}_{t+1}K_{t+1}^{\alpha - 1} + 1 - \delta)\\
        C_{t+1} &= C_t (\beta) (\alpha A_{t+1}^{1-\alpha} K_{t+1}^{\alpha - 1} + 1- \delta) 
    \end{align*}
    Since $k_t A_t = K_t$ and $c_t A_t = C_t$, we see that:
    \begin{align*}
        A_{t+1} C_{t+1} &= A_t c_t \beta (\alpha A_{t+1}^{1-\alpha} K_{t+1}^{\alpha - 1} + 1- \delta) \\
        c_{t+1} (1+\gamma) &= \beta c_t (\alpha A_{t+1}^{1-\alpha} K_{t+1}^{\alpha - 1} + 1- \delta) \\
        \frac{c_{t+1}}{c_t} &= \frac{(\alpha A_{t+1}^{1-\alpha} K_{t+1}^{\alpha - 1} + 1- \delta)}{(1+p)(1+\gamma)}
    \end{align*}
    Thus, we can see that if we let the approximation hold, we get the following;
    \[
    \frac{c_{t+1}}{c_t} = \alpha k^{\alpha - 1}_{t+1} + 1 - \delta - \gamma - p
    \]
\end{subsol}
\begin{subprob}
    
\end{subprob}
\begin{subsol}
    We can define the balanced steady state as the following, we want consumption and $k$ to stop growing, which implies that $\frac{c_{t+1}}{c_t} = 1$. Therefore, we can solve for the following:
    \begin{align*}
        1 &= \alpha k^{\alpha -1} + 1 - \delta - \gamma - \rho\\
        \gamma + \delta + \rho &= \alpha k^{\alpha -1} \\
        k &=\left( \frac{\alpha}{\gamma + \delta + \rho} \right)^\frac{1}{1-\alpha}
    \end{align*}
    and thus using the fact that $k_t^\alpha = c_t + k_{t+1}(\gamma + \delta) + g_t$, we find that
    \[
    c = \left( \frac{\alpha}{\gamma + \delta + \rho} \right)^\frac{\alpha}{1-\alpha} - \left( \frac{\alpha}{1\gamma + \delta + \rho} \right)^\frac{1}{1-\alpha}(\gamma + \delta) - g
    \]
\end{subsol}

\begin{subprob}
    
\end{subprob}
\begin{subsol}
    We see that in steady state, or when $\delta c_t = 0$, we find that $k =\left( \frac{\alpha}{\gamma + \delta + \rho} \right)^\frac{1}{1-\alpha}$, which implies that we can draw the following graph. 
    \begin{figure}[H]
        \centering
        \includegraphics[width=0.75\linewidth]{IMG_8EF9F0F00E27-1.jpeg}
        \caption{Graph for Question 2.5}
        \label{Question 2.5}
    \end{figure}
\end{subsol}
\begin{subprob}
    
\end{subprob}
\begin{subsol}
    If $\delta k = 0$, we find that:
    \begin{align*}
        k^\alpha - k(\gamma + \delta) &= c_t + g_t\\
        k^\alpha - k(\gamma + \delta) - g &= c_t\\
        \frac{\partial c_t}{\partial k} &= \alpha k^{\alpha - 1} - (\gamma + \delta)\\
        \frac{\partial^2 c_t}{\partial k^2} &= \alpha (\alpha -1 )k^{\alpha - 2} < 0
    \end{align*}  
    Since $\frac{\partial^2 c_t}{\partial k^2}$, the shape is indeed hump shaped, and has a peak. which can be derived as follows:
    \begin{align*}
        0 &= \alpha k^{\alpha -1} - (\gamma + \delta)\\
        \gamma + \delta &= \alpha k^{\alpha - 1}\\
        \frac{\gamma + \delta}{\alpha} &= k^{\alpha - 1}\\
        k &= \left( \frac{\alpha}{\gamma + \delta} \right)^\frac{1}{1-\alpha}
    \end{align*}
    \begin{figure}[H]
        \centering
        \includegraphics[width=0.75\linewidth]{IMG_77F6F09843C1-1.jpeg}
        \caption{Question for Question 2.6}
        \label{Question 2.6}
    \end{figure}
\end{subsol}

\begin{subprob}
    
\end{subprob}

\begin{subsol}
    Note that:
    \[
    k_{ss} = \left( \frac{\alpha}{\gamma + \delta + \rho} \right)^\frac{1}{1-\alpha} < \left( \frac{\alpha}{\gamma + \delta} \right)^\frac{1}{1-\alpha} = k_g
    \]
    which immplies that the steady state is indeed to the left. Intuively, this makes sense, as we have to factor in the discount rate of future consumption, $\rho$. The left hand of the inequality has a $\rho$ in it, indicating that we tend to consume more in the present because we discount consumption in future periods. The above is held with strict equality if $\rho = 0$, or where future and present consumption are the same to both individuals. \\
\end{subsol}

\noindent
\textbf{(8, 9 ,10)} \hspace{1cm}\\
We begin with a general analysis. Recall this equation, we can see that:
\[
    \frac{c_{t+1}}{c_t} = \alpha k^{\alpha - 1}_{t+1} + 1 - \delta - \gamma - \rho
\]
If $\Delta k > 0$, this implies that $\Delta c$ must increase and we are left to $k_{ss}$, which promotes a moving up of $c$. By symmetry, we can see that if $\Delta k < 0$, this implies that $\Delta c$ must decrease and we are to the right of $k_{ss}$, which promotes a moving down of $c$. Simiarly, if $\Delta c < 0$, we are in the bottom region of the hump, and thus, we see that $\Delta k$ must increase. And, if $\Delta c > 0$, we are in the upper region of the hump, and thus, we see that $\Delta k$ must decrease. We are left with the following diagram. 

\begin{figure}[H]
    \centering
    \includegraphics[width=0.75\linewidth]{MOVETHISPLZ.jpeg}
    \caption{Graph for 8,9,10}
    \label{fig:enter-label}
\end{figure}

for \textbf{8}, we see that $k_t \to 0$ and $c_t \to \infty$ and similarly, \textbf{9} we see that $k_t \to \infty$ and $c_t \ to 0$. \textbf{10}, see above figure. 

\setcounter{subproblem}{10}
\begin{subprob}
    
\end{subprob}

\begin{problem}{3}
    
\end{problem}

\newpage
\setcounter{subproblem}{0}
\begin{problem}{4}
    
\end{problem}
\begin{subprob}
    
\end{subprob}
\begin{subsol}
    The transversality condition is the following:
    \[
    \lim_{t \to \infty} \lambda_t K_t = \lim_{t \to \infty} \beta^t u'(c_t) K_t = 0
    \]
    The intution is that we want to maximize all of the resources given to us, with no room left to spend or to save. 
\end{subsol}
\begin{subprob}
    
\end{subprob}
\begin{subsol}
    At steady state, we can see that:
    \[
    f(k) - c -\delta k = 0 \iff c = f(k) - \delta k
    \]
    We what to maximize with respect to $c$, so we see that:
    \[
    \frac{\partial c}{\partial k} = f'(k) -\delta = 0 \implies f'(k) = \delta
    \]
    Thus, we see that:
    \[
    \delta = f'(k_{gr}) > f'(k)
    \]
\end{subsol}
\begin{subprob}
    
\end{subprob}
\begin{subsol}
    Consider the Household's maximization problem. We see that:
    \begin{align*}
        L = \max_{c_t, k_t}  \sum_{t = 0}^{\infty} u(c_t) + \lambda_t (k_{t+1} - f(k_t) -c_t + (1 - \delta)k_t) 
    \end{align*}
    with the following FoCs:
    \begin{align*}
        [c_t] & \quad \lambda_t = \beta^t u'(c_t)\\
        [k_t] & \quad \lambda_{t-1} = \lambda_t(f'(k_t) + 1 - \delta)
    \end{align*}
    From here, we can combine the above expressions to derive the Euler Equation, 
    \[
    u'(c_t) = \beta u'(c_{t+1})(f'(k_{t+1}) + 1 - \delta)
    \]
    Note the following:
    \begin{align*}
        u'(c_0) &= \beta u'(c_1)(f'(k_1) + 1 -\delta)\\
        &= \beta(\beta u'(c_2)(f'(k_2) + 1 - \delta))(f'(k_{1} + 1 - \delta))\\
        &\implies \beta^t u'(c_t) \Pi_{s=1}^t (f'(k_s) + 1 - \delta)
    \end{align*}
    Thus, we can see that 
    \[
    \beta^t u'(c_t) = u'(c_0) \Pi_{s=1}^t (f'(k_s) + 1 -\delta)^{-1}
    \]
    Therefore, 
    \[
    \lim_{t \to \infty} u'(c_0) \Pi_{s=1}^t (1+f'(k_s) - \delta)^{-1} k_{t+1} = 0
    \]
\end{subsol}
%%%%%%%%%%%%%%%%%%%%%%%%%%%%%%%%%%%%%
%Do not alter anything below this line.
\end{document}
