\documentclass[11pt]{article}
\usepackage{graphicx}
\usepackage[utf8]{inputenc}
\usepackage{geometry}
\usepackage{titlesec}
\usepackage{enumitem}
\usepackage{hyperref} 
\usepackage{amsmath}
\usepackage{ gensymb }
\usepackage{ amssymb }
\usepackage{float}
\usepackage{amsthm}
\usepackage{float}

\newcommand{\R}{\mathbb{R}}
\newcommand{\N}{\mathbb{N}}
\newcommand{\Q}{\mathbb{Q}}
\newcommand{\Z}{\mathbff{Z}}
\newcommand{\st}{\text{s.t}}

\title{PSET 3}
\author{Anthony Yoon}
\date{1/29/2025}
\begin{document}
\maketitle
\section{}
\subsection*{a}
We are interested in the following optimization problem:
\begin{align*}
    \max & \quad px^\frac{1}{3}_1x^\frac{1}{3}_2 - \omega_1x_1 - \omega_2x_2
\end{align*}
We see that the FOCs are 
\begin{align*}
    [x_1] & \quad \frac{1}{3} p x_1^{-\frac{2}{3}} x_2^\frac{1}{3} - \omega_1 \leq 0 \quad \text{for }  x_1 \geq 0 \\
    [x_2] & \quad \frac{1}{3} px_1^\frac{1}{3}x_2^{-\frac{2}{3}} - \omega_2 \leq 0 \quad \text{for } x_1 \geq 0
\end{align*}
We can see that $x_1, x_2, \neq 0$ as this would cause the FOCs to become undefined. From here, divide the FOCs to get the relation $\omega_1 x_1 = \omega_2 x_2$ Using this expression, we can substiute this into the FOCs to get that 
\[
x_1^* = \frac{p^3}{27\omega_1^2\omega_2} \quad x_2^* = \frac{p^3}{27\omega_1\omega_2^2}
\]
Therefore, we see that:
\[
y^* = \left( \frac{p^6}{3^6\omega_1^3\omega_2^3} \right)^\frac{1}{3} = \frac{p^2}{9\omega_1\omega_2}
\]
Thus, we see that 
\[
PF = py^* - \omega_1x_1^* - \omega_2x_2^* = p \left( \frac{p^2}{9\omega_1\omega_2} \right) - \omega_1 \frac{p^3}{27\omega_1^2\omega_2} - \omega_2\frac{p^3}{27\omega_1\omega_2^2} = \frac{p^3}{27\omega_1\omega_2}
\]
\subsection*{b}
We can see that for the IDFs
\[
\frac{\partial x_1^*}{\partial \omega_1} = -2 \left( \frac{p^3}{27\omega_2\omega_1^3} \right)
\]
and 
\[
\frac{\partial x_2^*}{\partial \omega_2} = -2 \left( \frac{p^3}{27\omega_1\omega_2^3} \right)
\]
Note that both quantities are bounded above by 0, as $p, \mathbf{\omega}$ are strictly positive. For the ODF, we see that 
\[
\frac{\partial y^*}{\partial p} = \frac{2p}{9\omega_1\omega_2}
\]
which is always positive for the same reasons. For the PF, note that 
\[
\frac{\partial \pi(\mathbf{\omega}, y)}{\partial p} = \frac{p^2}{9\omega_1\omega_2} > 0 \quad \frac{\partial \pi(\mathbf{\omega}, p)}{\partial \omega_1} = \frac{-p^3}{27\omega_1^2\omega_2} <0 \quad \frac{\partial \pi(\mathbf{\omega}, p)}{\partial \omega_2} = \frac{-p^3}{27\omega_1\omega_2^2} <0 
\]
\subsection*{c}
Proof that IDF is homogenous in degree 0, let $t >0$, we see that 
\[
x_1^*(t\mathbf{\omega}, tp) =\frac{(tp)^3}{27(t\omega_1)^2t\omega_2} = \frac{t^3 p^3}{27t^3\omega_1^2\omega^2} = \frac{p^3}{27\omega_1^2\omega_2} = x_1^*(\mathbf{\omega}, p)
\]
and similarly
\[
x_2^*(t\mathbf{\omega}, tp) = \frac{(tp)^3}{27t\omega_1(t\omega_2)^2} = \frac{t^3p^3}{27t^3\omega_1\omega_2^2} = \frac{p^3}{27\omega_1\omega_2^2} = x_2^*(\mathbf{\omega}, p)
\]
Proof that OSF is homogenous in degree 0, let $t > 0$, we see that 
\[
y^*(t\mathbf{\omega}, tp) = \frac{t^2p^2}{9t^2\omega_1\omega_2} = \frac{p^2}{9\omega_1\omega_2} = y^*(\mathbf{\omega}, p)
\]
Proof that PF is homogenous in degree 1, let $t > 0$, we see that 
\[
\pi(\mathbf{t\omega}, pt) = \frac{t^3p^3}{27t^2\omega_1\omega_2} = \frac{tp^3}{27\omega_1\omega_2} = t\pi(\mathbf{\omega}, p)
\]
\subsection*{d}
To see if Hotelling's Lemma holds, note that 
\[
\frac{\partial \pi(\mathbf{\omega}, p)}{\partial p} = \frac{3p^2}{27\omega_1\omega_2} = \frac{p^2}{9\omega_1\omega_2} = y^*
\]
and 
\[
\frac{\partial \pi}{\partial \omega_1} = \frac{-p^3}{27\omega_1^2\omega_2} = -x^*_1
\]
and 
\[
    \frac{\partial \pi}{\partial \omega_1} = \frac{-p^3}{27 \omega_1\omega_2^2} = -x^*_2
\]
\subsection*{e}
If $\alpha = \beta = 0.5$, this is a Cobb Douglas function. Thus, we can use the function derived from the notes to see that the FOCs yields:
\[
p = \frac{\omega_1^\frac{1}{2}\omega_2^\frac{1}{2}}{0.25(0.25)} = 16
\]
Thus, if price is less than 16, there has exists no solution as there is no profit to be found for any level of production. 
\subsection*{f}
If $\alpha + \beta = 1$, we see that we are left with the Cobb Douglas function where the fucntion can be only be derived based on external given factors $(\omega, p,  etc.)$. So if $\alpha + \beta \neq 1$, then we can derive a solution. 
\section{}
Note that cost is minized when $\alpha x_1  = \beta x_2 = y$. This is because we are working with a minimum function, a similar arguement of that to PSET 2 Q3b. This implies that 
\[
x_1^* = \frac{y}{\alpha} \quad x_2^* = \frac{y}{\beta}
\]
Therefore, we can see that with the given assumptions that 
\[
c(\omega, y) = \frac{y}{\alpha} + \frac{y}{\beta}
\]
Thus, we can see that we are interested in the following profit maxization problem:
\[
\max_y py - y\left(\frac{1}{\alpha} + \frac{1}{\beta}\right) = \max_y y \left( p - \left( \frac{1}{\alpha} + \frac{1}{\beta} \right) \right)
\]
which implies that profit is dependent on exogenously given parameters, or rather we are in the form of \emph{price times(output - input)}. Thus, we can see that for our OSF, which we can derive because we are soley focused on output versus input:
\[y^* = 
\begin{cases}
    \text{undefined} & p > \frac{1}{\alpha} + \frac{1}{\beta} \text{ as firms cannot have infinite output}\\
    [0, \infty) & p = \frac{1}{\alpha} + \frac{1}{\beta} \text{ as there is 0 profit}\\
    0 & p < \frac{1}{\alpha} + \frac{1}{\beta} \text{ as cost $>$ price}
\end{cases}
\]
Similarly, since IDF is dependent on the OSF, we can see that \textbf{similarly for the same reasons}
\[
x_1^* = \begin{cases}
    \text{undefined} &\quad p > \frac{1}{\alpha} + \frac{1}{\beta} \\
    \frac{y}{\alpha} &\quad p = \frac{1}{\alpha} + \frac{1}{\beta} \\
    0 &\quad p < \frac{1}{\alpha} + \frac{1}{\beta}
\end{cases}
\]
\[
x_2^* = \begin{cases}
    \text{undefined} &\quad p > \frac{1}{\alpha} + \frac{1}{\beta} \\
    \frac{y}{\beta}&\quad p = \frac{1}{\alpha} + \frac{1}{\beta} \\
    0 &\quad p < \frac{1}{\alpha} + \frac{1}{\beta}
\end{cases}
\] 
Thus, this implies that the profit function is:
\[
\pi^* = \begin{cases}
    \text{undefined} &\quad p > \frac{1}{\alpha} + \frac{1}{\beta} \\
    0 & \quad p \leq \frac{1}{\alpha} + \frac{1}{\beta}
\end{cases}
\]
\section{}
Note that the following production function, or the perfect subsitute production function:
\[
y = \alpha x_1 + \beta x_2
\]
we can exchange between $x_1$ and $x_2$. From here, we can see that cost will be minimized if we purchase only of the cheaper of the 2 goods. To prove that, we see that we are interested in the following cost minimization problem 
\begin{align*}
    \min & \quad \omega_1 x_1 + \omega_2 x_2\\
    \st & \quad y \leq \alpha x_1 + \beta x_2
\end{align*}
However, plugging the constraint into the object function yields the following optimization problem:
\[
\min_{x_1} \quad \omega_1 x_1 + \omega_2 \left( \frac{y - \alpha x_1}{\beta} \right)
\]
From here, we differeniate with respect to $x_1$, we can se that we get
\[
\omega_1 - \frac{\omega_2 \alpha}{\beta} 
\]
However, note that this quantity is dependent on parameters, so we can make the following deductions:
\begin{itemize}
    \item If $\omega_1 - \frac{\omega_2 \alpha}{\beta} < 0$ , we can see that increasing the input of $x_1$ will decrease cost, so we can see that in this case $x_1 = \frac{y}{\alpha}$ and $x_2 = 0$
    \item If $\omega_1 - \frac{\omega_2 \alpha}{\beta} > 0$, we can see that increasing the input of $x_1$ will increase cost, so we can see that in this case $x_2 = \frac{y}{\beta}$ and $x_1 = 0$
    \item If $\omega_1 = \frac{\omega_2 \alpha}{\beta}$, we can see that any input will give us the optimal amount. So this implies that $x_1 \in [0, \frac{y}{\alpha}]$ and $x_2 = \frac{y - \alpha x_1}{\beta}$
\end{itemize}
Thus, we can see that cost is minimized when we choose the minimum of the inputs, or rather 
\[
c(\omega, y) = \min \left\{  \frac{\omega_1 y}{\alpha}, \frac{\omega_2 y}{\beta}\right\}
\]
For notational sake, let us call $c(\omega, y) = C$ Note that we are know interetsed in the following profit maximaization problem:
\[
\max y (1 - C)
\]
So we can see that our ODF (for same reasons as 2)
\[
y^* = \begin{cases}
    \text{undefined} & \quad C < 1\\
    [0, \infty] & \quad C = 1\\
    0 & \quad C > 1
\end{cases}
\]
and using the proof above and let $W = \omega_1 - \frac{\omega_2 \alpha}{\beta}$, we can see that 
\[
x_1^* = \begin{cases}
    0 & \quad W < 0 \text{ and } C = 1\\
    [0, \frac{y}{\alpha}] & \quad W = 0 \text{ and } C = 1\\
    \frac{y}{\alpha} & \quad W > 1 \text{ and } C = 1\\
    \text{undefined} & \quad C < 1\\
    0 & \quad C > 1
\end{cases}
\]
\[
x_2^* = \begin{cases}
    0 & \quad W > 0 \text{ and } C = 1\\
    [0, \frac{y}{\beta}] & \quad W = 0 \text{ and } C = 1\\
    \frac{y}{\beta} & \quad W < 1 \text{ and } C = 1\\
    \text{undefined} & \quad C < 1\\
    0 & \quad C > 1
\end{cases}
\]
Thus, we can see our profit function is 
\[
\pi(\omega, p) = \begin{cases}
    \text{undefined} & C <1\\
    0 & C \geq 1
\end{cases}
\]
\end{document}