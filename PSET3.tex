\documentclass[11pt]{article}
\usepackage{graphicx}
\usepackage[utf8]{inputenc}
\usepackage{geometry}
\usepackage{titlesec}
\usepackage{enumitem}
\usepackage{hyperref} 
\usepackage{amsmath}
\usepackage{ gensymb }
\usepackage{ amssymb }
\usepackage{float}
\usepackage{amsthm}
\usepackage{float}

\newcommand{\R}{\mathbb{R}}
\newcommand{\N}{\mathbb{N}}
\newcommand{\Q}{\mathbb{Q}}
\newcommand{\Z}{\mathbff{Z}}
\newcommand{\st}{\text{s.t}}

\title{PSET 3}
\author{Anthony Yoon}
\date{1/29/2025}
\begin{document}
\maketitle
\section{}
\subsection*{a}
We are interested in the following optimization problem:
\begin{align*}
    \max & \quad px^\frac{1}{3}_1x^\frac{1}{3}_2 - \omega_1x_1 - \omega_2x_2
\end{align*}
We see that the FOCs are 
\begin{align*}
    [x_1] & \quad \frac{1}{3} p x_1^{-\frac{2}{3}} x_2^\frac{1}{3} - \omega_1 \leq 0 \quad \text{for }  x_1 \geq 0 \\
    [x_2] & \quad \frac{1}{3} px_1^\frac{1}{3}x_2^{-\frac{2}{3}} - \omega_2 \leq 0 \quad \text{for } x_1 \geq 0
\end{align*}
We can see that $x_1, x_2, \neq 0$ as this would cause the FOCs to become undefined. From here, divide the FOCs to get the relation $\omega_1 x_1 = \omega_2 x_2$ Using this expression, we can substiute this into the FOCs to get that 
\[
x_1^* = \frac{p^3}{27\omega_1^2\omega_2} \quad x_2^* = \frac{p^3}{27\omega_1\omega_2^2}
\]
Therefore, we see that:
\[
y^* = \left( \frac{p^6}{3^6\omega_1^3\omega_2^3} \right)^\frac{1}{3} = \frac{p^2}{9\omega_1\omega_2}
\]
Thus, we see that 
\[
PF = py^* - \omega_1x_1^* - \omega_2x_2^* = p \left( \frac{p^2}{9\omega_1\omega_2} \right) - \omega_1 \frac{p^3}{27\omega_1^2\omega_2} - \omega_2\frac{p^3}{27\omega_1\omega_2^2} = \frac{p^3}{27\omega_1\omega_2}
\]
\subsection*{b}
We can see that for the IDFs
\[
\frac{\partial x_1^*}{\partial \omega_1} = -2 \left( \frac{p^3}{27\omega_2\omega_1^3} \right)
\]
and 
\[
\frac{\partial x_2^*}{\partial \omega_2} = -2 \left( \frac{p^3}{27\omega_1\omega_2^3} \right)
\]
Note that both quantities are bounded above by 0, as $p, \mathbf{\omega}$ are strictly positive. For the ODF, we see that 
\[
\frac{\partial y^*}{\partial p} = \frac{2p}{9\omega_1\omega_2}
\]
which is always positive for the same reasons. For the PF, note that 
\[
\frac{\partial \pi(\mathbf{\omega}, y)}{\partial p} = \frac{p^2}{9\omega_1\omega_2} > 0 \quad \frac{\partial \pi(\mathbf{\omega}, p)}{\partial \omega_1} = \frac{-p^3}{27\omega_1^2\omega_2} <0 \quad \frac{\partial \pi(\mathbf{\omega}, p)}{\partial \omega_2} = \frac{-p^3}{27\omega_1\omega_2^2} <0 
\]
\subsection*{c}
Proof that IDF is homogenous in degree 0, let $t >0$, we see that 
\[
x_1^*(t\mathbf{\omega}, tp) =\frac{(tp)^3}{27(t\omega_1)^2t\omega_2} = \frac{t^3 p^3}{27t^3\omega_1^2\omega^2} = \frac{p^3}{27\omega_1^2\omega_2} = x_1^*(\mathbf{\omega}, p)
\]
and similarly
\[
x_2^*(t\mathbf{\omega}, tp) = \frac{(tp)^3}{27t\omega_1(t\omega_2)^2} = \frac{t^3p^3}{27t^3\omega_1\omega_2^2} = \frac{p^3}{27\omega_1\omega_2^2} = x_2^*(\mathbf{\omega}, p)
\]
Proof that OSF is homogenous in degree 0, let $t > 0$, we see that 
\[
y^*(t\mathbf{\omega}, tp) = \frac{t^2p^2}{9t^2\omega_1\omega_2} = \frac{p^2}{9\omega_1\omega_2} = y^*(\mathbf{\omega}, p)
\]
Proof that PF is homogenous in degree 1, let $t > 0$, we see that 
\[
\pi(\mathbf{t\omega}, pt) = \frac{t^3p^3}{27t^2\omega_1\omega_2} = \frac{tp^3}{27\omega_1\omega_2} = t\pi(\mathbf{\omega}, p)
\]
\subsection*{d}
To see if Hotelling's Lemma holds, note that 
\[
\frac{\partial \pi(\mathbf{\omega}, p)}{\partial p} = \frac{3p^2}{27\omega_1\omega_2} = \frac{p^2}{9\omega_1\omega_2} = y^*
\]
and 
\[
\frac{\partial \pi}{\partial \omega_1} = \frac{-p^3}{27\omega_1^2\omega_2} = -x^*_1
\]
and 
\[
    \frac{\partial \pi}{\partial \omega_1} = \frac{-p^3}{27 \omega_1\omega_2^2} = -x^*_2
\]
\subsection*{e}
\textbf{REDO}
If $\alpha = \beta = 0.5$, this is a Cobb Douglas function. Thus, we can use the function derived from the notes to see that the FOCs yields:
\[
p = \frac{\omega_1^\frac{1}{2}\omega_2^\frac{1}{2}}{0.25(0.25)} = 16
\]
Thus, if price is less than 16, there has exists no solution as there is no profit to be found for any level of production. 
\subsection*{f}
If $\alpha + \beta = 1$, we see that we are left with the Cobb Douglas function where the fucntion can be only be derived based on external given factors $(\omega, p,  etc.)$. So if $\alpha + \beta \neq 1$, then we can derive a solution. 
\section{}
Note that cost is minized when $\alpha x_1  = \beta x_2 = y$. This is because we are working with a minimum function, a similar arguement of that to PSET 2 Q3b. This implies that 
\[
x_1^* = \frac{y}{\alpha} \quad x_2^* = \frac{y}{\beta}
\]
Therefore, we can see that with the given assumptions that 
\[
c(\omega, y) = \frac{y}{\alpha} + \frac{y}{\beta}
\]
Thus, we can see that we are interested in the following profit maxization problem:
\[
\max_y py - y\left(\frac{1}{\alpha} + \frac{1}{\beta}\right) = \max_y y \left( p - \left( \frac{1}{\alpha} + \frac{1}{\beta} \right) \right)
\]
which implies that profit is dependent on exogenously given parameters, or rather we are in the form of \emph{price times(output - input)}. Thus, we can see that for our OSF, which we can derive because we are soley focused on output versus input:
\[y^* = 
\begin{cases}
    \text{undefined} & p > \frac{1}{\alpha} + \frac{1}{\beta} \text{ as firms cannot have infinite output}\\
    [0, \infty) & p = \frac{1}{\alpha} + \frac{1}{\beta} \text{ as there is 0 profit}\\
    0 & p < \frac{1}{\alpha} + \frac{1}{\beta} \text{ as cost $>$ price}
\end{cases}
\]
Similarly, since IDF is dependent on the OSF, we can see that \textbf{similarly for the same reasons}
\[
x_1^* = \begin{cases}
    \text{undefined} &\quad p > \frac{1}{\alpha} + \frac{1}{\beta} \\
    \frac{y}{\alpha} &\quad p = \frac{1}{\alpha} + \frac{1}{\beta} \\
    0 &\quad p < \frac{1}{\alpha} + \frac{1}{\beta}
\end{cases}
\]
\[
x_2^* = \begin{cases}
    \text{undefined} &\quad p > \frac{1}{\alpha} + \frac{1}{\beta} \\
    \frac{y}{\beta}&\quad p = \frac{1}{\alpha} + \frac{1}{\beta} \\
    0 &\quad p < \frac{1}{\alpha} + \frac{1}{\beta}
\end{cases}
\] 
Thus, this implies that the profit function is:
\[
\pi^* = \begin{cases}
    \text{undefined} &\quad p > \frac{1}{\alpha} + \frac{1}{\beta} \\
    0 & \quad p \leq \frac{1}{\alpha} + \frac{1}{\beta}
\end{cases}
\]
\section{}
Note that the following production function, or the perfect subsitute production function:
\[
y = \alpha x_1 + \beta x_2
\]
we can exchange between $x_1$ and $x_2$. From here, we can see that cost will be minimized if we purchase only of the cheaper of the 2 goods. To prove that, we see that we are interested in the following cost minimization problem 
\begin{align*}
    \min & \quad \omega_1 x_1 + \omega_2 x_2\\
    \st & \quad y \leq \alpha x_1 + \beta x_2
\end{align*}
However, plugging the constraint into the object function yields the following optimization problem:
\[
\min_{x_1} \quad \omega_1 x_1 + \omega_2 \left( \frac{y - \alpha x_1}{\beta} \right)
\]
From here, we differeniate with respect to $x_1$, we can se that we get
\[
\omega_1 - \frac{\omega_2 \alpha}{\beta} 
\]
However, note that this quantity is dependent on parameters, so we can make the following deductions:
\begin{itemize}
    \item If $\omega_1 - \frac{\omega_2 \alpha}{\beta} < 0$ , we can see that increasing the input of $x_1$ will decrease cost, so we can see that in this case $x_1 = \frac{y}{\alpha}$ and $x_2 = 0$
    \item If $\omega_1 - \frac{\omega_2 \alpha}{\beta} > 0$, we can see that increasing the input of $x_1$ will increase cost, so we can see that in this case $x_2 = \frac{y}{\beta}$ and $x_1 = 0$
    \item If $\omega_1 = \frac{\omega_2 \alpha}{\beta}$, we can see that any input will give us the optimal amount. So this implies that $x_1 \in [0, \frac{y}{\alpha}]$ and $x_2 = \frac{y - \alpha x_1}{\beta}$
\end{itemize}
Thus, we can see that cost is minimized when we choose the minimum of the inputs, or rather 
\[
c(\omega, y) = \min \left\{  \frac{\omega_1 y}{\alpha}, \frac{\omega_2 y}{\beta}\right\}
\]
For notational sake, let us call $c(\omega, y) = C$ Note that we are know interetsed in the following profit maximaization problem:
\[
\max y (1 - C)
\]
So we can see that our ODF (for same reasons as 2)
\[
y^* = \begin{cases}
    \text{undefined} & \quad C < 1\\
    [0, \infty] & \quad C = 1\\
    0 & \quad C > 1
\end{cases}
\]
and using the proof above and let $W = \omega_1 - \frac{\omega_2 \alpha}{\beta}$, we can see that 
\[
x_1^* = \begin{cases}
    0 & \quad W < 0 \text{ and } C = 1\\
    [0, \frac{y}{\alpha}] & \quad W = 0 \text{ and } C = 1\\
    \frac{y}{\alpha} & \quad W > 1 \text{ and } C = 1\\
    \text{undefined} & \quad C < 1\\
    0 & \quad C > 1
\end{cases}
\]
\[
x_2^* = \begin{cases}
    0 & \quad W > 0 \text{ and } C = 1\\
    [0, \frac{y}{\beta}] & \quad W = 0 \text{ and } C = 1\\
    \frac{y}{\beta} & \quad W < 1 \text{ and } C = 1\\
    \text{undefined} & \quad C < 1\\
    0 & \quad C > 1
\end{cases}
\]
Thus, we can see our profit function is 
\[
\pi(\omega, p) = \begin{cases}
    \text{undefined} & C <1\\
    0 & C \geq 1
\end{cases}
\]
\section{}
\subsection*{a}
If we fix $x_2 = 1$, we can see that 
\begin{align*}
    y = x^\alpha_1 \iff x_1^* = y^\frac{1}{\alpha}
\end{align*} 
This implies that 
\[
sc(\omega, y) = \omega_2 + \omega_1 y^\frac{1}{\alpha}
\]
We can see that if $\alpha = 1$, we get constant return to scale, and if $\alpha > 1$ we can see we get decreasing return to scale and $\alpha < 1$ we get increasing return to scale. Now we solve the profit maximaization case. where we want to 
\[
py - \omega_2 - \omega_1 y^\frac{1}{\alpha}
\]
Now, we are interested in the profit maximizing short run supply function. 
If we let $\alpha = 1$, we get the equation:
\[
py - \omega_2 - \omega_1 y \iff y(p - \omega_1) - \omega_2
\]
which implies that $\omega_2$ is a fixed cost that the company must pay for. Thus, we can see the following production functions.
\[
y^* = \begin{cases}
    \text{undefined} & \quad p > \omega_1\\
    [0,\infty) & \quad p = \omega_1\\
    0 & \quad p < \omega_1
\end{cases}
\]
If $\alpha > 1$, note the following. Let us look at the marginal cost. 
\[
\frac{\partial sc}{\partial y} = -\frac{\omega_2}{y^2} - \left( \frac{1-\alpha}{\alpha} \right)\omega_1 y^\frac{1- 2\alpha}{\alpha}
\]
Note that with $\alpha >1$, we can see that the above quantity is always negative, which implies that average cost is strictly less than that of marginal cost, which means the firm would want to produce infinite amounts. If $\alpha < 1$, we can see that this makes the cost term is convex, which implies that there exists a maxiumum and increasing marginal costs, thus this implies that when we solve the Langrangain, we get:
\[
y^* = \left( \frac{\alpha p}{\omega_1} \right)^\frac{\alpha}{1-\alpha}
\]
\subsection*{b}
We begin with $\alpha > 1$, we can see that in this case, $q^s$ and $q^d$ are undefined as $y^*$ itself tends to infinity. 


Let $\alpha = 1$, as before, we can see that our output demand is dependent on the relationship between $p$ and $\omega_1$. We can see that if $\omega_1 < p$ equilbrium is undefined because each firm wants to produce as much as possible. Similarly, if $\omega_1 > p$, we can see that the firms do not want to produce anything, but that has a mismatch in the the demand the consumers want, so there is no equilbrium here. If we let $\omega_1 = p$, we can see the following. 
\[
p = 10 - q \iff q = 10 - p \iff q = 10 - \omega_1
\]
Additionally, note that both firms are indifferent to producing any level outout. Hence, we can let $y$ be any value. So we can see that:
\[
q^s = 2y = 10 - p = 10 - \omega_1
\]
this implies that $y = \frac{10 - \omega_1}{2}$, so we can see that our equilbrium price is $\omega_1$ When we analyze the profits of each firm, note that:
\[
\pi = py - \omega_1 y - \omega_2 = - \omega_2
\]
where $\omega_2$ represents the sunk cost of producing. 


Let $\alpha < 1$. We already know that our level of output is fixed, where we can see that 
\[
q^s = 2y^* = 2 \left( \frac{\alpha p}{\omega_1} \right)^\frac{\alpha}{1-\alpha}
\]
and thus when we let $p^E$ denote equilbrium price:
\[
10 - p^E = 2 \left( \frac{\alpha p^E}{\omega_1} \right)^\frac{\alpha}{1-\alpha}
\]
Let us assume that there exists a solution to this function, and we will still denote this as $p^E$. Now, let us consider the profit of the firms:
\[
\pi = y(p^E - \omega_1) - \omega_2 = \left( \frac{\alpha p^E}{\omega_1} \right)^\frac{\alpha}{1-\alpha} ( p^E - \omega_1) - \omega_2
\]
\subsection*{c}
Now, we impose the zero profit condition. Assume that every firm has the same production capabilities. From a previous PSET, we know that the value of $\alpha + \beta$ has significant impact on the Cobb Douglas production function. Using resuls derived from previous PSETs (2.2), we get the following:
\[
x_1 = \left( y \left( \frac{\beta}{\alpha} \right)^\beta \right)^\frac{1}{\alpha + \beta} x_2 = \left( y \left( \frac{\beta}{\alpha} \right)^\alpha \right)^\frac{1}{\alpha + \beta}
\] 
\[
c(\omega, y) = y^\frac{1}{\alpha + \beta} \left( \left( \frac{\beta}{\alpha} \right)^\frac{\alpha}{\alpha + \beta} + \left( \frac{\alpha}{\beta} \right)^\frac{\beta}{\alpha + \beta} \right)
\]
Let $A =  \left( \left( \frac{\beta}{\alpha} \right)^\frac{\alpha}{\alpha + \beta} + \left( \frac{\alpha}{\beta} \right)^\frac{\beta}{\alpha + \beta} \right)$. Now, we are interested in the profit maximaization problem, where 
\[
\max_y py - y^\frac{1}{\alpha + \beta} A
\]
If $\alpha + \beta > 1$, this means that the profit can just keep producing in order to maximize profit, incentivizing infinite productin, so the equilbrium does not exist here. \\


If $\alpha + \beta = 1$, we can see that we get $(p - A)y$. Thus, we can see that we have to concern the relationship between $p$ and $A$. So we get the following output supply functions for the same reasons for the previous problems. 
\[
y^* = \begin{cases}
    \text{undefined} & \quad p > A\\
    [0, \infty) & \quad p = A\\
    0 & \quad p < A
\end{cases}
\]
Thus, this implies that equilbrium only exists $p=A$. Therefore, we can see that $p =  A$ is our equilbrium price, and thus we are only interested in solving for the supply that would sastify this
\[
q^d = q^s  \iff 10 - A = 2y \iff y = \frac{10 - A}{2}
\]
If $\alpha + \beta < 1$, we have a decreasing return to scale problem. Given the optimization problem, let us differentiate with respect $y$, where we see that we get (assuming interior solutions):
\[
p = \frac{1}{\alpha + \beta} y^{\frac{1}{\alpha + \beta} - 1} A
\]
After some algebra, we get that:
\[
y = \left(\left( \frac{\alpha + \beta}{A} \right) p\right)^\frac{\alpha + \beta}{1-\alpha-\beta}
\]
By the zero profit condition, we can see that:
\[
\pi = py -y^\frac{1}{\alpha + \beta} A = p^\frac{1}{1-\alpha - \beta} \left( \left( \frac{\alpha + \beta}{A} \right)^\frac{\alpha + \beta}{1-\alpha - \beta} - \left( \frac{\alpha + \beta}{A} \right)^\frac{1}{1-\alpha-beta}\right) = 0
\]
This implies that $p =0$, which means no equilbrium exists. 
\section{}
\subsection*{a}
If $x_2$ is to be fixed at 1. 
\section{}
\subsection*{a}
If $x_1 = f$, then for both firms, 
\[
f(x_1, x_2) = 2x_2
\]
Thus our short term cost function is 
\[
sc = \frac{y}{2} \omega_2 + \omega_1 f = 0.5y + f
\]
Thus the profit function is:
\[
\pi = y(p - \frac{1}{2}) - f
\]
So note the following:
\[
y = \begin{cases}
    \text{undefined} & \quad p > 0.5\\
    [0,\infty) & \quad p = 0.5\\
    0 & \quad p < 0.5
\end{cases}
\]
Thus  if we let the equilbrium price be 0.5, we see that 
\[
2y = q^s = q^d = 10- p
\]
This implies that $q^d = q^s = 9.5$, which implies our profit is $-f$, a sunk cost. 
\subsection*{b}
In pursuit of a profit maximaization solution, the firm can choose not to produce. Thus, this means that they choose to produce nothing, where $y = 0$. However, this would cause equilbrium to not exist. 
\subsection*{c}
A similar situation happens. Firms profit maximize, which comes at $y = 0$, and thus equilbrium cannot be established as $q^d = 0$ 
\subsection*{d}
\subsection*{i}
We can note that setting this price floor $p < 0.5$, as noted in $a$, firms don't produce anything leading to no equilbrium. If $p = 0.5$, we can see that we reach the equilbrium stated in $a$. Similarly, if $p > 0.5$, we see that firms will want to produce an infininte amount, which causes the solution to be undefined.
\subsection*{ii}
If $s <f$, firms are not incentived to produce so both short and long run equilbrium cannot be reached. If $s \geq f$, we can see that firms will outweight their fixed cost, and thus produce as if nothing happened in $a$. So for the short run, we can see that if $s \geq f$, we have our solution that we derived $p^E = 0.5$, $q^d = q^s = 9.5$, $\pi^1 > 0, \pi^2 > 0$. Long run, if $s = f$, we hit the 0 profit condition, as the firm is compensated for their sunk cost. Therefore, note that $q^s = 9.5$ as calculated before hand as no firm can exit or enter the market due to the 0 profit condition. Thus, this implies that 
\[
q^d = q^s = 9.5 \quad \pi^1 = \pi^2 = 0
\] 
for $s = f$. If $s > f$, we can see that we are interested in the optimization problem:
\[
\max_y y(p-\frac{1}{2}) - f + s
\]
which implies that for 0 profit to hold, $p = \frac{f-s}{y} + 0.5$ must be true. From here, we can find that using the market demand if $ p = 10 - q^d$, this implies that 




\end{document}