\documentclass[11pt]{article}
\usepackage{graphicx}
\usepackage[utf8]{inputenc}
\usepackage{geometry}
\usepackage{titlesec}
\usepackage{enumitem}
\usepackage{hyperref} 
\usepackage{amsmath}
\usepackage{ gensymb }
\usepackage{ amssymb }
\usepackage{float}
\usepackage{amsthm}
\usepackage{float}

\newcommand{\R}{\mathbb{R}}
\newcommand{\N}{\mathbb{N}}
\newcommand{\Q}{\mathbb{Q}}
\newcommand{\Z}{\mathbff{Z}}
\newcommand{\st}{\text{s.t}}

\title{PSET4}
\author{Anthony Yoon}
\date{2/20/2025}
\begin{document}
\maketitle
\section{}
\subsection*{a}
We can first begin calulating MRS for each individual. Without a loss of generality, we can begin to note the following:
\[
MRS_i = \frac{\alpha (x_1^i)^{\alpha -1}(x_2^i)^\beta}{\beta(x_2^i)^{\beta - 1}(x_1^i)^\alpha}
\]
This above quantity implies that $\frac{x_2^1}{x_1^1} = \frac{x_2^2}{x_1^2}$. Let $e_1 = x_1^1 + x_1^2$ and $e_2 = x_2^1 + x_2^2$. Using this equalities as well as the implication derived from the MRSes, we can find that we get:
\[
x_2^1 = \frac{e_2}{e_1} x_1^1 \quad x_2^2 = \frac{e_2}{e_1} x_1^2
\]
Since both the above quantities are linear in nature with no intercept, this implies that indeed the contact curve is that of connecting endpoints. 
\subsection*{b}
Since we we are working with different utility functions, we can find that after similar caluclations to above that:
\[
MRS_1 = \frac{\alpha (x_1^1)^{\alpha - 1}(x_2^1)^{1-\alpha}}{(1-\alpha)(x_1^1)^\alpha (x_2^1)^{-\alpha}} = \frac{\alpha x_2^1}{(1-\alpha)x_1^1}
\]
and using similar calculations, we find that:
\[
MRS_2 = \frac{\beta x^2_2}{(1-\beta)x_1^2}
\]
Since we know that $1 > \alpha > \beta > 0$, we find that:
\[
\frac{\alpha}{1-\alpha} > \frac{\beta}{1-\beta}
\]
Thus, we can see that for MRS to equal to each other, we know that:
\[
\frac{x_2^1}{x_1^1} < \frac{x^2_2}{x^2_1}
\]
Using the equations derived above, we can find that:
\[
x_2^1 < \frac{e_2}{e_1} x_1^1
\]
this implies that the graph still intersects the origins, but now $x_2^1 < x_1^1$, where we have all a curve that will be strictly below that of the original line derived in \textbf{a}. The contract curve is seen as below. 
\subsection*{c}
For the contract curve to exist, we want $MRS_1 = MRS_2$. Let $e_1 = x_1^1 + x_1^2$. We can see that
\[
MRS_1 = MRS_2 \implies \alpha (x_1^1)^{\alpha - 1} = \beta (x_1^2)^{\beta - 1}
\]
Thus, substuting the endowment, we find that:
\[
    \alpha(x_1^1)^{\alpha - 1} = \beta (e_1 - x_1^1)^{\beta - 1}
\]
So we see that as $ x_1^1 \to e^1_1$, we find that the consumers will not consume any $x_2$, and consume only $x_1$ However, since we know that $\alpha > \beta$, this implies that consumer $x_1^1$ has greater value on $x_1^1$, which implies that $x_1^1  > x_1^2$. Note that any level of $x_2$ sastifies the MRS equality arguement, thus we are only concerned about when $e_1 = x_1^1 + x_1^2$ and one of these inputs are the utility maximizing solution. The contract curve will like the one below:
\section{}
\begin{figure}[H]
    \centering
    \includegraphics[width=0.75\linewidth]{econ20110_2a.jpeg}
    \caption{Graph for Question 2}
    \label{fig:1l}
\end{figure}
\subsection*{a}
See Graph 
\subsection*{b}
Given in Question \textbf{1}
\subsection*{c}
By definition, it is the contract curve between the curves. 
\subsection*{d}
The core would just be the point $(e_1^1, e_2^1)$ as any trade would make this individual worse off, hence no trade would not be blocked by any coalition. 
\subsection*{e}
We first derive the general Marshallian Demand function for each individual. We are interested in the following optimization problem:
\begin{align*}
    \max & \quad x_1x_2\\
    \st & \quad p_1e_1 + p_2e_2 \geq p_1x_1 + p_2x_2
\end{align*}
Where the langrangian is:
\[
L = x_1x_2 - \lambda(p_1e_1 + p_2e_2 - p_1x_1 - p_2x_2)
\]
wherw we see that the FOCs are:
\begin{align*}
    [x_1] &\quad x_2 + \lambda p_1 \leq 0 \text{ and } x_1 \geq 0\\
    [x_2] & \quad x_1 + \lambda p_2 \leq 0 \text{ and } x_2 \geq 0\\
    [\lambda] &\quad p_1e_1 + p_2e_2 \leq p_1 x_1 + p_2x_2 
\end{align*}
We can see all FOCs must be strict equality, as if that is not the case, then markets will fail to clear $([x_1], [x_2])$ and by the AU assumption that we want to use all of our endowment to maximize utility. Using these FOCs, we can find that:
\[
p_1x_1 = p_2x_2
\]
Thus, using this equation and the constraint, we find that:
\[
x_1^m = \frac{e_1}{2} + \frac{p_2}{2p_1}e_2 \quad x_2^m = \frac{p_1}{2p_2}e_1 + \frac{e_2}{2}
\]
Thus, we find that:
\[
x_1^1 = \frac{p_2}{2p_1} + 1 \quad x_2^1 = \frac{p_1}{p_2} + \frac{1}{2}
\]
\[
x_1^2 = \frac{3p_2}{2p_1} + 1 \quad x_2^2 = \frac{p_1}{p_2} + \frac{3}{2}
\]
Thus we can find that where we let $\mathbf{p} = (p_1, p_2)$:
\[
z_1(\mathbf{p}) = 2 + 2 \left( \frac{p_2}{p_1} \right)  - 4 =  - 2 + 2 \left( \frac{p_2}{p_1} \right) 
\]
\[
z_2(\mathbf{p}) = 2 + 2 \left( \frac{p_1}{p_2} \right)  - 4 =  - 2 + 2 \left( \frac{p_1}{p_2} \right) 
\]
and we can verify that:
\[
\mathbf{p} \cdot \mathbf{z} = \begin{bmatrix}
    p_1 & p_2 
\end{bmatrix} \cdot 
\begin{bmatrix}
    2 \left( \frac{p_2}{p_1} \right) - 2\\ 
    2 \left( \frac{p_1}{p_2} \right) - 2
\end{bmatrix} = 2p_2 - 2p_1 + 2p_1 - 2p_2 = 0
\]
\subsection*{f}
We can see if $p_2 = p_1$, we find that obviously, $\mathbf{z}$ goes to 0. Thus, we find that the set of Walrasian equlibria is $p^*(\mathcal{E}) = \{p_1, p_2) | p_1 = p_2\}$ We find that a Walrasian equilbrium allocation is
\[
x^W =\{ (x_1, x_2)\} = \{ (1.5, 1.5), (2.5, 2.5)\}
\]
since there is one relative price, we can find that the set of Walrasian Equilbirum Allocations is: 
\[
W(\mathcal{E}) = \bigcup_{p^*} x^W(p^*(\mathcal{E}), \mathcal{E})  = \{ (1.5, 1.5), (2.5, 2.5)\}
\]
\subsection*{g}
Note that $e_1 = e_2$, this implies that the slope of the line is 1, which means that all allocations that have $x_1 = x_2$ will be in the core. Thus, we can see that the above set is a subset of the core. 
\subsection*{h}
Switching the utility function, we find that we are interested in the following optimization problem:
\begin{align*}
    \max & \quad x_1^{\frac{2}{3}}x_2^{\frac{1}{3}}\\
    \st & \quad p_1e_1 + p_2e_2 \leq p_1 x_1 + p_2x_2 
\end{align*}
the constraint remains the same, but with the following FOCs. 
\begin{align*}
    [x_1] & \quad \left( \frac{2}{3} x_1^{-\frac{1}{3}} \right)x_2^\frac{1}{3} = p_1 \lambda\\
    [x_2] & \quad \left( \frac{1}{3}x_1^{-\frac{1}{3}} \right)x_1^\frac{2}{3} = p_2 \lambda
\end{align*}
We know that these FOCs must have strict equality due to the same reasons as stated above. Using these FOCs, the following can be derived:
\[
2x_2p_2 = x_1p_1
\]
which implies that:
\[
x_1^2 = \frac{2}{3p_1} (p_1e_1 + p_2e_2) \quad x_2^2 = \frac{1}{3p_1}(p_1e_1 + p_2e_2)
\]
Using, previous results, we can find that:
\[
z_1(\mathbf{p}) = 1 + \frac{p_2}{2p_1} + \frac{4}{3} + \frac{2p_2}{p_1} - 4 = \frac{5p_2}{2p_1} - \frac{5}{3}
\]
\[
z_2(\mathbf{p}) = \frac{1}{2} + \frac{p_1}{p_2} + \frac{2p_1}{3p_2} + 1 -4 = \frac{5p_1}{3p_2} - \frac{5}{2}
\]
Note that this implies that $3p_2 = 2 p_1$, as this is the only relative price that makes Walras' law hold. Thus, we see that:
\[
x^W = \{(x_1, x_2)\} = \left\{ \left( \frac{4}{3}, 2 \right), \left( \frac{8}{3}, 2 \right)\right\}
\]
and since we have only one relative price, we can find that the set of Walrasian equilbirum is 
\[
    W(\mathcal{E}) = \bigcup_{p^*} x^W(p^*(\mathcal{E}), \mathcal{E})  =  \left\{ \left( \frac{4}{3}, 2 \right), \left( \frac{8}{3}, 2 \right)\right\}
\]
\subsection*{i}
Note that the optimal values of $x_1$ and $x_2$ are only dependent on prices, we are interested in solving the cases purely dependent on making the value of the endowments equal to each other. 
\subsubsection*{First Economy}
since we know that $p_1 = p_2$, and we are interested in equating market values to each other, we should know that:
\[
p_1(e_1^1 + e_2^1) = p_2(e_1^2 + p_2^2) \implies (e_1^1 + e_2^1) = (e_1^2 + p_2^2)
\]
Let $T = (T_1, T_2)$ where each $T_i$ represents the transfer from each individual 1 to individual 2 in respect to each good, we can see that:
\[
2 - T_1 + 1 - T_2 = 2 + T_1 + 3 + T_2 
\]
However, since prices are equal, we can solve with respect with each good. Thus, we can see that:
\[
2 - T_1 = 2 + T_1 \iff T_1 = 0 \quad 1 - T_2 = 3 + T_2 \iff T_2 = -1
\]
Thus, $T = (0, -1)$, which means that no transfer of good is done, but one $x_1$ is transfered from person 2 to person 1, with leads to:
\[
\mathbf{e^1} = \mathbf{e^2}= (2, 2)
\]
\subsubsection*{Second Economy}
Now. we see that $3p_2 = 2p_1$, using a similar logic to that above and same definition of $T$, we see that:
\begin{align*}
    p_1(2-T_1) + p_2(1-T_2) &= p_1(2+T_1) + p_2(3 + T_1)\\
    \frac{3p_2}{2}(2 - T_1) + p_2(1 - T_2) &= \frac{3p_2}{2}(2+T_1) + p_2(3+T_1)\\
\end{align*}
Using a similar arguement, we find that $T = (0, -1)$, which means that we obtain the same individual endowment as before.
\section{}
\begin{figure}[H]
    \centering
    \includegraphics[width=0.75\linewidth]{econ20110_3a.jpeg}
    \caption{Graph for 3a - 3d}
    \label{fig:aentaesras-aslasasabel}
\end{figure}
\subsection*{a}
See Graph
\subsection*{b}
Note that equating the MRSes here is not ideal due to the given utility functions, so a more logical arguement must be used. We can see that in order for each consumer to maximize their utility, consumer 1 and 2 should have only have good 1 and 2 respectively. Thus, the only pareto optimal points would be the bottom right corners of the edgeworth box, where we would draw the indfference curves as seen as below, making sure that consumer's 1 indifference curve goes through the bottom right corner of the indifference point and consumer's 2 indifference curve goes thorugh the top left corner of the Edgeworth Box. A graph can be seen above, where the red line is the contract curve. 
\subsection*{c}
The core would be characterized as $\{e^1, e^2\} = \{(2, 0), (0, 2) \}$, as moving away from this allocation will cause a coalition to reject it. 
\subsection*{d}
Since the Walrasian Equilbrium is a subset of the core, we find that that the only allocation in Walrasian Equilbirum is:
\[
x^W(p^*(\mathcal{E}), \mathcal{E}) = \{(2, 0), (0, 2) \}
\]
Thus, we see that: 
\[
W(\mathcal{E}) = \bigcup_{p^*}(p^*(\mathcal{E}), \mathcal{E}) = \{(2, 0), (0, 2) \}
\]
\subsection*{e}
\begin{figure}[H]
    \centering
    \includegraphics[width=0.75\linewidth]{econ20110_3.jpeg}
    \caption{Graph for 3e}
    \label{fig:enter-label}
\end{figure}
\subsubsection*{a}
See above
\subsubsection*{b}
Note that each consumer maximize utilty if they consume only good $x_1$ and $x_2$ respectively. Thus, we see that the contract curve follows. 
\subsubsection*{c}
Same logic in the previous exchange economy, which yields the same answer as above. 
\subsubsection*{d}
Same logic in the previous exchange economy, which yields the same answer as above. 
\subsection*{f}
We can consider the 3 cases, $\alpha = 1, \alpha < 1, \alpha > 1$. 
\begin{itemize}
    \item If $\alpha > 1$ we are left with the case above, as consumer 1 and 2 value $x_1$ and $x_2$ more heavily as we see respectively. 
    \item If $\alpha < 1$ we see that the order of preference between $x_1, x_2$ is changed between consumer 1 and 2. Thus, we would get a very similar result, but we would have to swap the Walrsian equilbrium, which yields the result: 
    \[
        W(\mathcal{E}) = \bigcup_{p^*}(p^*(\mathcal{E}), \mathcal{E}) = \{(0, 2), (2, 0)\} 
    \]
    and we would simply reflect our core and contract curve over the line that connects the bottom left and top right corners of the edgeworth box. 
    \item If $\alpha = 1$, we see that both consumers value $x_1$ and $x_2$ equally. Thus, we are left the graph seen in \textbf{2}
\end{itemize}
\section{}
\subsection*{a}
We can note that as we are all using the minimum function, we can see that $x_1^1 = x_2^1 = x_2^2 = x_3^2 = x_1^3 = x_3^3$, which implies that we are interested in the allocation:
\[
    x_1^1 = x_2^1 = x_2^2 = x_3^2 = x_1^3 = x_3^3 = 0.5
\]
which is optimal as moving away would cause utilty lose and maximizes utlity for all and markets clear. Thus, this allocation is in the core, and let this allocation equal $\overline{x}$
\subsection*{b}
Consider the UMP for consumer 1. We are interested in the following optimization problem:
\begin{align*}
    \max & \quad \min\{x_1^1, x_2^1\}\\
    \st & \quad x_1^1 p_1 + p_1x_2^1 \leq p e_1
\end{align*}
note that we want $x_1^1 = x_2^1$, which implies using the strict equality of the constraint (which we know is true by the AU assumption) that 
\[
x_1^1 = x_2^1 = \frac{p_1}{p_1 + p_2}
\]
using a similar logic, we can see that:
\[
x_2^2 = x_3^2 = \frac{p_2}{p_2+p_3}
\]
and
\[
x_1^3 = x_3^3 = \frac{p_3}{p_1+p_3}
\]
Thus, we can now analyze the aggegate demand functions. 
We find that $z_i = x_i^1 + x_i^2 + x_i^3 - 1$ where $i \in \{1,2,3\}$. After some algebra and using the Marshallian demand functions taht we derived, we find that $p_1 = p_2 = p_3$. This implies that:
\[
    x_1^1 = x_2^1 = x_2^2 = x_3^2 = x_1^3 = x_3^3 = 0.5
\]
Thus, we can see that $p^*(\mathcal{E}) = \{(p_1, p_2, p_3) \in \R^3 | p_1 = p_2 = p_3\}$. Thus, we can see tha $x^W(p^*(\mathcal{E}), \mathcal{E}) = \{ (0.5,0.5,0), (0, 0.5, 0.5), (0.5, 0, 0.5)\}$ Since we are working with only one relative price, we can note that 
\[
    W(\mathcal{E}) = \bigcup_{p^*} x^W(p^*(\mathcal{E}), \mathcal{E}) = \{ (0.5,0.5,0), (0, 0.5, 0.5), (0.5, 0, 0.5)\}
\]
\section*{6}
\subsection*{a}
We first solve for the firms profits maximization problem, as the profit firm profit produces will affect the budget of the consumers. Thus, if $f(h_1, h_2) = h_1 + h_2$, using results from a previous homework, we can find that the cost minimization problem is:
\[
c(\omega_1, \omega_2, y) = y\min\{\omega_1, \omega_2\}
\]
and let $C = \min(\omega_1, \omega_2)$. Thus, we can see that we have the following:
\begin{align*}
    y^* &= \begin{cases}
        \infty & p > C\\
        [0,\infty) & p = C\\
        0 & p < C
    \end{cases}\\
    h_1^* &= \begin{cases}
        \infty & p>C\\
        [0, y] & p = C\\
        0 & p < C
    \end{cases}\\
    h_2^* &= \begin{cases}
        \infty & p > C\\
        y - h_2 & p = C\\
        0 & p > C
    \end{cases}\\
    \pi &= \begin{cases}
        \infty & p > C\\
        0 & p \leq C
    \end{cases}
\end{align*}
Note that only possible case where firms produce and don't shut down is when $p = C$, as in any other scenario firms shut down and don't produce anything. Now we are interested in solving the consumer's UMP problem. We are interested in the following optimization problem for each individual:
\begin{align*}
    \max & \quad l x\\
    \st & \quad px + \omega l \leq \omega T \\
    \text{where} & \quad l + h = 24 = T
\end{align*}
FOCs are as follows:
\begin{align*}
    [l] & \quad x - \omega \lambda \leq \text{ and } l \geq 0\\
    [x] & \quad l - p \lambda \leq 0 \text{ and } x \geq 0\\
    [\lambda] & \quad px + \omega l = 24 \omega 
\end{align*}
we can use these FOCs to find that:
\[
px = l\omega  \quad x^* = \frac{12\omega}{p} \quad l^* = 12
\]
which thus this implies that $h^* = 12$ Thus, we can see that in any market, as long as $p = C(\omega_1, \omega_2)$, then markets clear. Thus, we want a variable amount of $h_1, h_2$, which implies that $p = \omega_1 = \omega_2$. Thus, we see that:
\[
p^*(\mathcal{E}) = \{p \in \R \hspace{2pt} | \hspace{2pt} p = \omega_1 = \omega_2\}
\]
\subsection*{b}
We can see that:
\[
    W(\mathcal{E}) = \bigcup_{p^*} x^W(p^*(\mathcal{E}), \mathcal{E}) = \{(h_1, h_2, x_1, x_2, l_1, l_2, y) = (12,12,12,12,12, 12 ,24)\}
\]
as $x_i = 12\frac{\omega_i}{p} = 12$. Since both utility functions are strictly increasing, we can see that this is an unique solution and it exists. 
\subsection*{c}
Note that we have doubled the firms producitivity, which implies that we have a similar solution to the cost minimization problem, so we can see that:
\[
c_2(\omega_1, \omega_2, y) = \frac{1}{2}\min \{ \omega_1, \omega_2\} = \frac{1}{2}c_1(\omega_1, \omega_2, y)
\]
So we can note that we can follow a very similar logic to that above, where we can see that we still  get very similar solutions where we let $C_2 = 0.5C$:
\begin{align*}
    y^* &= \begin{cases}
        \infty & p > C_2\\
        [0,\infty) & p = C_2\\
        0 & p < C_2
    \end{cases}\\
    h_1^* &= \begin{cases}
        \infty & p>C_2\\
        [0, y] & p = C_2\\
        0 & p < C_2
    \end{cases}\\
    h_2^* &= \begin{cases}
        \infty & p > C_2\\
        y - h_2 & p = C_2\\
        0 & p > C_2
    \end{cases}\\
    \pi &= \begin{cases}
        \infty & p > C_2\\
        0 & p \leq C_2
    \end{cases}
\end{align*}
We can also see that our UMP for each consumer is the same, so we get the exact same results, where we see that $l^* = h^* = 12$ and similarly we can see that $x_i = 12\frac{\omega_1}{p}$. We can see that by a simiar logic to above, we see that $p = 0.5\omega_1 = 0.5\omega_2$, and thus:
\[
p^*(\mathcal{E}) = \{ p \in \R \hspace{2pt} | \hspace{2pt} p = 0.5\omega_1 = 0.5\omega_1\}
\]
thus, we see that:
\[
W(\mathcal{E}) = \bigcup_{p^*} x^W(p^*(\mathcal{E}), \mathcal{E}) = \{(h_1, h_2, x_1, x_2, l_1, l_2, y) = (12,12,24,24,12, 12 ,48)\}
\]
\subsection*{d}
We can see that we now have 2 firms, and we want to make Walrasian equilbrium exist. Note that firm 2 is a more efficient version of firm 1
\end{document}
