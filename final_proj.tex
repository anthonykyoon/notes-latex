\documentclass[12pt]{article}
\usepackage[a4paper,margin=1in]{geometry}
\usepackage{setspace}  % For spacing
\usepackage{titlesec}  % For section formatting
\usepackage{graphicx}  % For figures
\usepackage{amsmath}   % For mathematical symbols
\usepackage{hyperref}  % For clickable links
\usepackage{newtxtext}  % Improved Times New Roman font
\usepackage{float}
\usepackage{listings}
\usepackage{hyperref}
\usepackage{bm}
\usepackage{booktabs}  % Improves table formatting
\usepackage[table]{xcolor}

\titleformat{\section}{\large\bfseries}{\thesection}{1em}{}

\title{Econ Research Project}
\author{Chimin Liu, Anthony Yoon, Allison Yang\thanks{We would like to thank Professor Fang for his advice and guidance on this project}}
\date{\today}

% \doublespacing

\begin{document}
\maketitle
\tableofcontents
\newpage
\section{Introduction}
With over 7,000 students, the University of Chicago’s Undergraduate College provides education to numerous students. Due to the on-campus housing requirement, the College also provides ample room and board opportunities for students to take advantage of so that students can socialize with one another during their underclassmen years. This on-campus requirement has also helped foster UChicago’s coveted “House culture,” where students go on field trips and participate in bonding activities together in the dorms, providing them with a reliable social network from the day they set foot on campus.


However, once students are released from the on-campus housing requirement, many (54\% in 2014) choose to opt out of the dorms for off-campus apartments around Hyde Park. UChicago could benefit from having students pay for on-campus housing to develop community building between lowerclassmen and upperclassmen, to optimize the success of students who have easier access to classroom building and study spaces, and to have a steady revenue stream to fund housing operations, dining services, campus maintenance, on top of higher occupancy rates in the rented dorm building, Woodlawn Residential Commons. 


Inspired by the many variables that motivate students to leave the dorms as well as UChicago’s preference to have students stay, we chose to examine incentives that would lead UChicago students to stay in the dorms for our research project


In this paper, we treat UChicago as the firm and undergraduate students as the consumer. Specifically, UChicago wants to produce occupancy, or optimize for the number of students staying on campus. The students (the consumers) are choosing to maximize their utility by consuming one of either good, on or off campus housing. In this paper, we ask the question: \textbf{How can we model the incentives UChicago wishes to provide to motivate students to stay in on-campus housing using the Cournot Model and an annoymous game?}
\section{Associated Costs of Living}
We begin with a perspective of the various costs of living on and off campus. According to the \href{https://financialaid.uchicago.edu/undergraduate/how-aid-works/undergraduate-costs/}{University of Chicago Finacial Aid Website} and \href{https://web.archive.org/web/20160304084033/http://colleges.usnews.rankingsandreviews.com/best-colleges/university-of-chicago-1774/student-life}{the US News report in 2014}, the costs and the quantity of people are as follows: 
\begin{table}[H]
    \centering
    \begin{tabular}{c|c|c}
        & On campus & Off Campus\\
        \hline
        Room and Board & \$20109 & \$17502\\ 
        \hline
        Num. of People & 4133 & 3520\\
        \hline
    \end{tabular}
\end{table}
Under the assumption that there are approximately 1913 students per grade level, we see that there are approximately 307 upperclassmen that choose to stay on campus. This indicates that the majority of upperclassmen choose to live off campus. Thus, we can estimate that assuming that these numbers have remained constant throughout time that majority of upperclassmen choose to live off campus. 
\section{Assumptions of the model}
We first we would like to some assumptions. Mainly being,
\begin{itemize}
    \item When the university lacks the space to house students, the University will convert increasing the number of people living in a room to accomodate for the students. The students will be strictly worse off if this happens. 
    \item The improvements and determinets to quality of life that a student recieves from a university policy is equal. 
\end{itemize}
\section{Model}
\subsection{Modeling student's choice to move on or off campus}
Given that we know that floor of upperclasmen living on campus is $\phi$ and the ceiling $\xi$, we now consider the following game. We define the players of the game as the upperclassmen at the University, or 
\[
Players = \{1, 2, \dots, N \}
\]
and with the actions to move off or on campus, or denoted as follows:
\[
Actions = \{F, O\}
\]
with the base payoffs for eah individual as follows:
\[
U_i(F) = \Lambda \quad U_i(O) = \gamma
\]
for $i \in players$. Note that $\Lambda > \gamma$ as the price for living off campus is cheaper than that of living on campus. 
Let $K$ equal the improvement or detrition to the standard of living. To collect the number of individuals living on and off campus, the University sends outs the 2025-2026 housing application for individuals to fill out. 


However, depending on the number of upperclassmen staying on campus, the University has actions that it can take. Using the bounds that were derived before, the following cases emerge:
\begin{enumerate}
    \item If less than or equal $\phi$ studenst stay on campus, the University is worse off compared to last year. Thus, to compensate for this, the quality of life is reduced on campus. Thus, the payoffs can be defined as \begin{itemize}
        \item $U_i(O)$: $\gamma - K$
        \item $U_i(F)$: $\Lambda - K$
    \end{itemize}
    \item If the number of classmen  $ \in (\phi,\xi)$ students stay on campus, the school has not hit their target number of upperclassmen, and to incentive more upperclassmen to stay off campus, they increase the quality of life for only those living on campus. Thus, the payoffs can be defined as \begin{itemize}
        \item $U_i(O)$: $\gamma + K$
        \item $U_i(F)$: $\Lambda$
    \end{itemize}
    \item If there are exactly $\xi$ upperclassmen stay on campus, the University's target has been hit, and thus, they have enough budget and he ability to raise the quality of life across the board. Thus, the payoffs can be defined as \begin{itemize}
        \item $U_i(O)$: $\gamma + K$
        \item $U_i(F)$: $\Lambda + K$
    \end{itemize}
    \item If there are more than $\xi$ upperclassmen, the university crams students into dorms by adding more and the quality of living lowers. Thus, the payoffs can be defined as \begin{itemize}
        \item $U_i(O)$: $\gamma - K$
        \item $U_i(F)$: $\Lambda$
    \end{itemize}
\end{enumerate}
\subsection{Solving for optimal incentives}
The university aims to have $\xi$ number of upperclassmen living on campus. To simplify notation, let $x$ be the number of upperclassmen who choose the live on campus. Thus, the incentive, or $K$, that the university provides must have only one feasible nash equilbria, such that $\xi$ number of upperclassmen choose to live on campus. Thus, we must derive some relationship between $\Lambda, \phi, K$ to enforce this notion. To simplify notation, let $x$ be the number of upperclassmen who choose the live on campus. We claim that the condition that enforces this is $k > \Lambda - \gamma$. 
\subsubsection{Casework}
\subsubsection*{1. $\mathbf{x = 0}$}
Since everyone is living off campus, moving to on campus would make them strictly worse off, thus, this makes this case a Nash Equilbria. 
\subsubsection*{2. $\mathbf{0 < x <} \hspace{1pt} \bm{\phi}$}
Consider the possible deviations:
\begin{align*}
    u_i(O) = \gamma - K \quad &\to \quad u_i(F) = \Lambda - K \\
    u_i(F) = \Lambda - K\quad &\to \quad u_i(O) = \gamma + K
\end{align*}

\subsubsection*{3. $\mathbf{n = \xi - 1}$} 
Note that in this case, there is no profitable deviation, as:
\[
U_i(O) =  \quad \to \quad U_i(O) = 
\]
\subsubsection*{4. $\mathbf{x = \xi}$}
\subsubsection*{5. $\mathbf{x = \xi + 1}$}
\subsubsection*{6. $\mathbf{x = \phi}$}
\subsubsection*{7. $\mathbf{x = \phi + 1}$}
\subsubsection*{8. $\mathbf{\phi + 2 < x < \xi -1}$}
\subsubsection*{9. $x > \xi + 1$}
\subsubsection*{10. $\mathbf{n = N}$}
Consider the following deviation:
\[
U_i(O) = \gamma - K \quad \to \quad  U_i(F) = \Lambda 
\]
we see that $k > \gamma - \Lambda$ makes this not a profitable deviation. 


\end{document}