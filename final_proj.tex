\documentclass[12pt]{article}
\usepackage[a4paper,margin=1in]{geometry}
\usepackage{setspace}  % For spacing
\usepackage{titlesec}  % For section formatting
\usepackage{graphicx}  % For figures
\usepackage{amsmath}   % For mathematical symbols
\usepackage{hyperref}  % For clickable links
\usepackage{newtxtext}  % Improved Times New Roman font
\usepackage{float}
\usepackage{listings}
\usepackage{hyperref}
\usepackage{setspace}
\usepackage{bm}
\usepackage{amsfonts}
\usepackage{booktabs}  % Improves table formatting
\usepackage[table]{xcolor}

\newcommand{\R}{\mathbb{R}}
\newcommand{\N}{\mathbb{N}}
\newcommand{\Q}{\mathbb{Q}}
\newcommand{\Z}{\mathbff{Z}}
\newcommand{\st}{\text{s.t}}

\onehalfspacing
\titleformat{\section}{\large\bfseries}{\thesection}{1em}{}

\title{Econ Research Project}
\author{Chimin Liu, Anthony Yoon, Allison Yang\thanks{We would like to thank Professor Fang for his advice and guidance on this project}}
\date{\today}


\begin{document}
\maketitle
\tableofcontents
\newpage
\section{Introduction}
With over 7,000 students, the University of Chicago’s Undergraduate College provides education to numerous students. Due to the on-campus housing requirement, the College also provides ample room and board opportunities for students to take advantage of so that students can socialize with one another during their underclassmen years. This on-campus requirement has also helped foster UChicago’s coveted “House culture,” where students go on field trips and participate in bonding activities together in the dorms, providing them with a reliable social network from the day they set foot on campus.


However, once students are released from the on-campus housing requirement, many (54\% in 2014) choose to opt out of the dorms for off-campus apartments around Hyde Park. UChicago could benefit from having students pay for on-campus housing to develop community building between lowerclassmen and upperclassmen, to optimize the success of students who have easier access to classroom building and study spaces, and to have a steady revenue stream to fund housing operations, dining services, campus maintenance, on top of higher occupancy rates in the rented dorm building, Woodlawn Residential Commons. 


Inspired by the many variables that motivate students to leave the dorms as well as UChicago’s preference to have students stay, we chose to examine incentives that would lead UChicago students to stay in the dorms for our research project


In this paper, we treat UChicago as the firm and undergraduate students as the consumer. Specifically, UChicago wants to optimize for the number of students staying on campus. The students (the consumers) are choosing to maximize their utility by consuming one of either good, on or off campus housing. In this paper, we ask the question: \textbf{How can we model the incentives UChicago wishes to provide to motivate students to stay in on-campus housing using an annoymous game with externalities?}
\section{Associated Costs of Living}
We begin with a perspective of the various costs of living on and off campus. According to the \href{https://financialaid.uchicago.edu/undergraduate/how-aid-works/undergraduate-costs/}{University of Chicago Finacial Aid Website} and \href{https://web.archive.org/web/20160304084033/http://colleges.usnews.rankingsandreviews.com/best-colleges/university-of-chicago-1774/student-life}{the US News report in 2014}, the costs and the quantity of people are as follows: 
\begin{table}[H]
    \centering
    \begin{tabular}{c|c|c}
        & On campus & Off Campus\\
        \hline
        Room and Board & \$20109 & \$17502\\ 
        \hline
        Num. of People & 4133 & 3520\\
        \hline
    \end{tabular}
\end{table}
Under the assumption that there are approximately 1913 students per grade level, we see that there are approximately 307 upperclassmen that choose to stay on campus. This indicates that the majority of upperclassmen choose to live off campus. 
We also assume that there are about 100 occupancies per house, and with 48 houses within the dorms, that there are approximately 4800 spots for students to live on campus. This means that currently the university has about 670 open spot thorughout a given school year. This implies that there is a wastage of capital, and thus a potental loss of profit. 
Thus, we can estimate that assuming that these numbers have remained constant throughout time that majority of upperclassmen choose to live off campus. 
\section{Assumptions of the model}
We first we would like to some assumptions. Mainly being,
\begin{itemize}
    \item The number of occupancies in the dormitories are bounded by some constant $M \in \R^+$ where given our estimates for the new academic year, $M = 4800$
    \item When the demand for rooms is greater than $M$, the university will increase each rooms'  occupancy by adding beds, making all students living on campus worse off. 
    \item The incentives we propose in our model is symmetric such that the improvements and detriments to the quality of life that a student recieves is constant for all students.
    \item We assume that the university is strictly better off the more rooms that they fill. Hence,  all occupancies being filled would be the optimal outcome. 
    \item All assumptions presented in class will hold in this analysis
\end{itemize}
\section{Model}
\subsection{Modeling student's choice to move on or off campus}
Given that we know the current number of upperclassmen staying on campus is suboptimal, the university decides to introduce a housing game before the next academic year in the hopes of increasing the upperclassmen's on-campus occupancy to the optimal level. Let the status quo number of upperclasmen living on campus be the bound $\phi$ and the optimal number of upperclassmen $\xi$. Utilizing the estimates from earlier, $\phi$ and $\xi$ must be in $[0,M]$ where $\xi = M$. Now consider the following one-shot game. We define the players of the game as the upperclassmen at the University.
\[
Players = \{1, 2, \dots, N \}
\]
and with the actions to live off or on campus, or denoted as follows:
\[
Actions = \{F, O\}
\]
with the base payoffs in utility for each individual as follows:
\[
U_i(F) = \Lambda \quad U_i(O) = \gamma
\]
for $i \in players$. Note that $\Lambda > \gamma$ as the price for living off campus is cheaper than that of living on campus. 

The university sends out the 2025-2026 housing application to upperclassmen with complete information of standard of living will change contingent on number of upperclassmen who live on campus. Let $K$ equal the improvement or detrition to the standard of living. 

Based on this information, the upperclassmen choose whether to live on or off campus.
However, depending on the number of upperclassmen staying on campus, the University has actions that it can take. Using the bounds that were derived before, the following cases emerge:
\begin{enumerate}
    \item If less than or equal $\phi$ studenst stay on campus, the University is worse off compared to last year. Thus, to compensate for this, the quality of life is reduced on campus. Thus, the payoffs can be defined as \begin{itemize}
        \item $U_i(O)$: $\gamma - K$
        \item $U_i(F)$: $\Lambda - K$
    \end{itemize}
    \item If the number of classmen  $ \in (\phi,\xi)$ students stay on campus, the school has not hit their target number of upperclassmen, and to incentive more upperclassmen to stay off campus, they increase the quality of life for only those living on campus. Thus, the payoffs can be defined as \begin{itemize}
        \item $U_i(O)$: $\gamma + K$
        \item $U_i(F)$: $\Lambda$
    \end{itemize}
    \item If there are exactly $\xi$ upperclassmen stay on campus, the University's target has been hit, and thus, they have enough budget and he ability to raise the quality of life across the board. Thus, the payoffs can be defined as \begin{itemize}
        \item $U_i(O)$: $\gamma + K$
        \item $U_i(F)$: $\Lambda + K$
    \end{itemize}
    \item If there are more than $\xi$ upperclassmen, the university crams students into dorms by adding more beds and the quality of living lowers. Thus, the payoffs can be defined as \begin{itemize}
        \item $U_i(O)$: $\gamma - K$
        \item $U_i(F)$: $\Lambda$
    \end{itemize}
\end{enumerate}
\subsection{Solving for optimal incentives}
The university aims to have $\xi$ number of upperclassmen living on campus. To simplify notation, let $x$ be the number of upperclassmen who choose the live on campus. Thus, the incentive, or $K$, that the university provides must have only one feasible nash equilbria, such that $\xi$ number of upperclassmen choose to live on campus. Thus, we must derive some relationship between $\Lambda, \phi, K$ to enforce this notion. To do so, consider the following. If $x = \xi$, and denoting unilateral deviation as the following:
\begin{align*}
    U_i(O) = \gamma + K \quad &\to \quad U_i(F) = \Lambda\\
    U_i(F) = \Lambda + K \quad &\to \quad U_i(O) = \gamma - K
\end{align*}
Note that if we want no profitable deviation, we want the following conditions to hold:
\begin{align*}
    n &\leq \gamma + k \iff k \geq \Lambda - \gamma > 0 \\
    \gamma - K & \leq \Lambda + K \iff 2k \geq \gamma - \Lambda
\end{align*}
Note that the second constraint is strictly negative, which implies that if $K \geq \Lambda - \gamma$, then the scenario when $\xi = x$ will be a nash equilbrium. To increase the strength of this relation, consider strict inquality. Thus, we claim that $K > \Lambda - \gamma$ is the relationship between $K, \Lambda, \gamma$, and we check for any other Nash Equilbria. 
\subsubsection{Casework}
\subsubsection*{1. $\mathbf{x = 0}$}
Since everyone is living off campus, moving to on campus would make them strictly worse off, thus, this makes this case a Nash Equilbria. 
\subsubsection*{2. $\mathbf{0 < x <} \hspace{1pt} \bm{\phi}$}
Consider the possible deviations:
\begin{align*}
    u_i(O) = \gamma - K \quad &\to \quad u_i(F) = \Lambda - K \\
    u_i(F) = \Lambda - K\quad &\to \quad u_i(O) = \gamma - K
\end{align*}
We see that $\Lambda - k > \gamma - k$ or $\gamma - k > \Lambda - K$ must hold to force a profitable deviation. Thus, we see that the first condition is always true, which implies that in this scenario, there is no Nash Equilbrium, regardless of $K$
\subsubsection*{3. $\mathbf{n = \xi - 1}$} 
To force a profitable deviation, 
consider all the possible deviations:
\begin{align*}
    u_i(O) = \gamma + K \quad &\to \quad u_i(F) = \Lambda \\
    u_i(F) = \Lambda \quad &\to \quad u_i(O) = \gamma + K
\end{align*}
This implies that either $\Lambda > \gamma + k$ or $\gamma + k > \Lambda$ forces a profitable deviation, which the second condition matches our proposed condition. 
\subsubsection*{4. $\mathbf{x = \xi}$}
If this is the case, we see that our claimed condition forces this to be a Nash equilbrium. 
\subsubsection*{5. $\mathbf{x = \xi + 1}$}
Consider the possible deviations:
\begin{align*}
    u_i(O) = \gamma - K \quad &\to \quad u_i(F) = \Lambda + K \\
    u_i(F) = \Lambda \quad &\to \quad u_i(O) = \gamma - K
\end{align*}
We see that for a profitable deviation to exist either $2k > \gamma - \Lambda$ or $k < \gamma - \Lambda$, which the first one is true from our given condition. Thus, this is not a Nash equilbrium. 
\subsubsection*{6. $\mathbf{x = \phi}$}
Consider the possible deviations:
\begin{align*}
    u_i(O) = \gamma - K \quad &\to \quad u_i(F) = \Lambda - K \\
    u_i(F) = \Lambda - K\quad &\to \quad u_i(O) = \gamma + K
\end{align*}
To force a profitable deviation, we need that $\Lambda - K > \gamma - K$ or $2k > \Lambda - \gamma$, which the first condition is always true from how $\Lambda$ and $\gamma$ are defined. 
\subsubsection*{7. $\mathbf{x = \phi + 1}$}
Consider the possible deviations:
\begin{align*}
    u_i(O) = \gamma + K \quad &\to \quad u_i(F) = \Lambda - K \\
    u_i(F) = \Lambda \quad &\to \quad u_i(O) = \gamma + K
\end{align*}
We see that this is very similar to $2k < \Lambda - \gamma$ and $k > \Lambda - \gamma$, of which the latter term matches our proposed condition. 
\subsubsection*{8. $\mathbf{\phi + 2 < x < \xi -1}$}
Consider the possible deviations:
\begin{align*}
    u_i(O) = \gamma + K \quad &\to \quad u_i(F) = \Lambda\\
    u_i(F) = \Lambda \quad &\to \quad u_i(O) = \gamma + K
\end{align*}
Note that $k > \Lambda - \gamma$ forces a profitable deviation. 
\subsubsection*{9. $x > \xi + 1$}
Consider the possible deviations:
\begin{align*}
    u_i(O) = \gamma - K \quad &\to \quad u_i(F) = \Lambda  \\
    u_i(F) = \Lambda \quad &\to \quad u_i(O) = \gamma - K
\end{align*}
Note that our proposed condition causes this to not be a nash equilbirum. 
\subsubsection*{10. $\mathbf{n = N}$}
Consider the following deviation:
\[
U_i(O) = \gamma - K \quad \to \quad  U_i(F) = \Lambda 
\]
Note that our proposed condition causes this to not be a nash equilbirum. 
\subsection{Analysis of casework}
We can see that there are 2 Nash equilbria, $x = 0$ and $x = \xi$. However, some attention must be brought to the case where $x = 0$, as this is a case that must be avoided. This is becasue this is the worst possible case for the University, but we can also show that this is the worst possible case for all the upperclassmen as well. Consider the aggregated utility of the case where $x = 0$, we see that 
\[
\text{total utility in 0 case} = N(\Lambda - K)
\]
and similarly when $x = \xi$
\[
\text{total utilty in } \xi \text{ case} = \xi(\gamma + K) + (N - \xi)(\lambda + K)
\]
we see that the total utility in the $x = \xi$ is clearly higher than that in the $x = 0$ case. Thus, it is also in the students' interest to move away from the $x = 0$. Thus, if there is some collective action among the upperclassmen or some additional university policy that encourages movement away from the $x = 0$ case.
\end{document}