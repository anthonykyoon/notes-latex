\documentclass[12pt]{article}
\usepackage[a4paper,margin=1in]{geometry}
\usepackage{setspace}  % For spacing
\usepackage{titlesec}  % For section formatting
\usepackage{graphicx}  % For figures
\usepackage{amsmath}   % For mathematical symbols
\usepackage{hyperref}  % For clickable links
\usepackage{newtxtext}  % Improved Times New Roman font
\usepackage{float}
\usepackage{listings}
\usepackage{booktabs}  % Improves table formatting
\usepackage[table]{xcolor}

\titleformat{\section}{\large\bfseries}{\thesection}{1em}{}

\title{Econ Research Project}
\author{Chimin Liu, Anthony Yoon, Allison Yang\thanks{We would like to thank Professor Fang for his advice on this project}}
\date{\today}


\begin{document}
\maketitle
\tableofcontents
\newpage
\section{Introduction}
With over 7,000 students, the University of Chicago’s Undergraduate College provides education to numerous students. Due to the on-campus housing requirement, the College also provides ample room and board opportunities for students to take advantage of so that students can socialize with one another during their underclassmen years. This on-campus requirement has also helped foster UChicago’s coveted “House culture,” where students go on field trips and participate in bonding activities together in the dorms, providing them with a reliable social network from the day they set foot on campus.


However, once students are released from the on-campus housing requirement, many (54\% in 2014) choose to opt out of the dorms for off-campus apartments around Hyde Park. UChicago could benefit from having students pay for on-campus housing to develop community building between lowerclassmen and upperclassmen, to optimize the success of students who have easier access to classroom building and study spaces, and to have a steady revenue stream to fund housing operations, dining services, campus maintenance, on top of higher occupancy rates in the rented dorm building, Woodlawn Residential Commons. 


Inspired by the many variables that motivate students to leave the dorms as well as UChicago’s preference to have students stay, we chose to examine incentives that would lead UChicago students to stay in the dorms for our research project


In this paper, we treat UChicago as the firm and undergraduate students as the consumer. Specifically, UChicago wants to produce occupancy, or optimize for the number of students staying on campus. The students (the consumers) are choosing to maximize their utility by consuming one of either good, on or off campus housing. In this paper, we ask the question: \textbf{How can we model the incentives UChicago wishes to provide to motivate students to stay in on-campus housing using Partial Equilbrium and the collective action problem?}
\section{Assumptions of the model}
We first we would like to introduce the assumptions of the models 
\end{document}