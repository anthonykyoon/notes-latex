\documentclass[11pt]{article}

% --- PACKAGES ---
\usepackage[utf8]{inputenc}
\usepackage{amsmath, amssymb, amsthm}
\usepackage{graphicx}
\usepackage{tikz}
\usepackage{enumitem}
\usepackage{fancyhdr}
\usepackage{geometry}
\usepackage{xcolor}
\usepackage{titlesec}
\usepackage{hyperref}
\usepackage[colorinlistoftodos]{todonotes}

% --- PAGE SETUP ---
\geometry{margin=1in}
\pagestyle{fancy}
\fancyhf{}
\rhead{ECON 20210}
\lhead{Anthony Yoon}
\rfoot{\thepage}

% --- SECTION FORMAT ---
\titleformat{\section}{\normalfont\Large\bfseries}{\thesection}{1em}{}
\titleformat{\subsection}{\normalfont\large\bfseries}{\thesubsection}{1em}{}
\titleformat{\subsubsection}[runin]{\normalfont\bfseries}{\thesubsubsection}{0.5em}{}[.]

% --- CUSTOM ENVIRONMENTS ---
\newtheorem{definition}{Definition}[section]
\newtheorem{theorem}{Theorem}[section]
\newtheorem{proposition}{Proposition}[section]
\newtheorem{example}{Example}[section]
\newtheorem{note}{Note}[section]

\newenvironment{econitemize}{
  \begin{itemize}[leftmargin=1.5em]
    \setlength\itemsep{0.4em}
}{\end{itemize}}

% --- COLORS ---
\definecolor{econblue}{RGB}{0,76,153}
\hypersetup{
    colorlinks=true,
    linkcolor=econblue,
    urlcolor=econblue,
    citecolor=econblue
}

% --- TITLE ---
\title{\textbf{ECON 20210 Notes}}
\author{Anthony Yoon}
\date{\today}

\begin{document}

\maketitle
\tableofcontents
\listoftodos
\newpage


\section{Lecture 1}
In Macroeconomics, we want to look at the economy in the whole, hence the "macro" aspect of the economy. In this field. we try to analyze the "trend" within the economy. For example. we may be interested in how the economy reacts when we give the population a stimulus check. However, we cannot be always sure that the previous trends will be representative of that of the current trends. 


However, when we think about the economy as a whole, there are a lot of data we need to work with. Hence, we can collapse the data into aggregate values, which are as follows:
\begin{itemize}
    \item GDP (Gross Domestic Product): How well is the local economy doing?
    \item GNP (Gross National Product): How well are the nationals of a an economy doing?
    \item Unemployment: How is the labor market functioning?
    \item Inflation: How much money do you have to have now to buy the same basket of goods you bought in 2000. 
    \item Stock price: How valuable are corporations. 
    \item CPI (Consumer Price index): Measure of the general goods and prices. Done by keeping track of a certain collection of goods. 
\end{itemize}
We can see that these are a lot easier to work with, rather than working with high-dimensional data. 
\subsection{GDP explained}
In it of itself, GDP is a flow of money. Specfically, the \emph{final} dollar amount produced per unit of time. We can measure GDP in mainly 3 ways. All goods and services purchased, produced, and all income earned. All of these have to add up to the same value, and if the product is unsold, then they are treated as self-bought goods. Usually, that value is the price. However, prices change over time, and we thus have to account of the temporal nature of prices. Thus, we have the following indicators of GDP, where we let $P$ indicate quantity and $Q$ denote the quantity of good, and $t$ be time, and $i$ be the index of the good:
\begin{itemize}
    \item $Y^n_t$ or the nominal GDP, values products at their \emph{current} dollars at a time $t$. \[
    Y^n_t = \sum_i P_{i,j}Q_{i,t}
    \]
    \item $Y^r_t$ or the real GDP, values products at a \emph{constant} dollars at time 0:\[Y^r_t = \sum_i P_{i,0} Q_{i,t}\]
    \item $P_t$: GDP Deflator (price index), which serves as a baseline comparison between the year of interest and the baseline year. 
    \begin{align*}
        P_t &= \frac{Y^n_t}{Y^r_t} \times 100 \\
        &= \frac{\sum_i P_{i,j}Q_{i,t}}{\sum_i P_{i,0} Q_{i,t}} \approx \frac{P_t}{P_0}
    \end{align*}
\end{itemize}
\todo{May need to but in chain weighted GDP here later.}
\subsection{Expenditure}
We can calculate expenditure, the total amount of money spent, as the following:
\[
Y = C + I + G + EX - IM 
\]
where 
\begin{itemize}
    \item $C$ denotes the consumption madde by Households
    \item $I$ denotes the physical investment, purchases of new capital goods by businesses. 
    \item $G$ Government expenditures (purchases and investments) us all expenditure for all levels of government. However, we exclude transfer payments. 
    \item $EX$ denotes the cost of exports
    \item $IM$ denotes the cost of imports\footnote{$NX$ denotes net exports, which is just $EX - IM$}
\end{itemize}
And we set this equal to the income, as noted by our assumption above, or 
\[
Y = C + I + G + EX - IM = wL + \pi rK + T
\]
where 
\begin{itemize}
    \item $wL$ denotes the wage and compensation to workers
    \item $rK$ denotes the compensation to capital owners
    \item $\pi$ denotes the corporate profits
    \item $T$ denotes Taxess
\end{itemize}
we also have the notion of production functions, which are very similar to that but \textbf{ECON 20210}, where 
\[
y = f(A,K, L, X)
\]
where
\begin{itemize}
    \item $A$ denotes technology 
    \item $K$ denotes capital stock 
    \item $L$ denotes labor
    \item $X$ denotes aother factors of production
\end{itemize}
\subsubsection{Criticisms of GDP as a measure of Economic Well-being}
One of the good things aout GDP is that GDP per capita is often correlated with measures as infact mortality rate (-), life expectancy (+), and literacy rate (+). However, GDP does not include non-market activities such as household production and other activities, as leisure is an important to one's well being.  
\subsection{Level versus Growth}
However, note that GDP $(X)$ is a time dependent variable. Hence, we should be interested in \emph{growth} as well as the \emph{level} of the good. If the we can describe the growth in a \emph{discrete} manner, we can see that:
\[
\gamma = \frac{X_{t+1} - X_t}{X_t} \quad X_{t+1} = (1+\gamma)X_t
\]
if we have a continous and exponentially growing GDP, we have the following:
\[
\lim_{n \to \infty} \left( 1 + \frac{\gamma}{n} \right)^n
\]
and since $e = \lim_{n \to \infty} \left( 1 + \frac{1}{n} \right)^n$, we see that:
\[
    \lim_{n \to \infty} \left( 1 + \frac{\gamma}{n} \right)^n = e^\gamma \implies X_t = X_0 \cdot e^{\gamma t} \iff \ln X_t = \ln X_0 + \gamma t
\]
This implies that if we take an contionus GDP, take the natural log of GDP with respect time and it is linear, then the GDP rate of growth is constant. 
\section{Lecture 2}
\subsection{CPI}
Inflation is usally measured by tracking the price of a basket of goods, hence we are not interested in the nominal or the real discussion of GDP. There are many ways to track inflation in the following ways:
\begin{itemize}
    \item GDP Deflator: Basket of goods and services produced domestically
    \item Consumer Price Index: Basket of goods and services consumed by househoulds
    \item Personal Consumption Expenditure (PCE) price index: Chained the prices togther, which yields to higher coverage. 
    \item Producer Price Index: Basket of goods purchased by producers. 
\end{itemize}
To calcluate the CPI, we can do the following. We fix the basket of goods, say $Q_{i,0}$, and we track the price throughout time, like the following:
\[
X_t = \sum_i P_{i, t} Q_{i,0}
\]
We can calculate the price index as taking the expenditure of a year and and comparing it to the base year. Or: 
\[
P_t = \frac{X_t}{X_0} \times 100
\]
Inflation is calcluated as the change in price index, or
\[
\pi_t = \frac{P_{t+1} - P_t}{P_t}
\]
Inflation can be expressed as the arithmetic mean of indvidual inflation rates weighted by the expenditure share, or rather:
\[
\frac{P_{i,t} Q_{i,0}}{\text{Expenditure}} \cdot \frac{P_{i,0}}{P_{i,0}} \implies \frac{P_{i,t}}{P_{i,0}}
\]
We can also calcualte the GDP deflator as:
\[
\frac{\text{Nomial GDP}}{\text{Real GDP}} = \frac{\sum P_{i,t}Q_{i,t}}{\sum P_{0,t}Q_{i,t}}
\]
CPI tends to be greater than of the GDP deflator. The CPI that we have been familar with until this point has been referred to as the \textbf{Lasperes} index which tends to over estimate the true cost of livin, as 
\[
c(u_o, P_t) \leq \sum p_{i,t} q_{i,0} \implies \frac{\sum_i p_{i,t}q_{i,0}}{\sum_i p_{i,0}, q_{i,0}} \geq \frac{C(u_0, P_t)}{C(u_0, P_0)}
\]
because the hicksian demand function will always produce the amount with the lowest cost. 
We can also refer to the \textbf{Passche} index, which is where:
\[
\frac{\sum P_{i,t} Q_{i,t}}{\sum P_{i,0} Q_{i,t}}
\]
which tends to underestimate the rate of inflation for $u_1$ as \todo{Write in proof here}
We can also view price index as the cost of achieving the same level of utility across multiple time periods. We can see that we can use the Hicksian Demand Functions, $C(u,P)$ where $u$ denotes utility and $P$ denotes the price vector. Thus, we can define the true inflation rate as 
\[
X_t = \frac{C(u_0, P_t)}{C(u_0, P_0)}
\]
and also note that the changing of the price vector can be seen as a substitution efffect of the preference of the goods. We can also define the \textbf{Fisher index} as the geometric average of these two indices.
\subsection{Proof that Lasperes index at base time approximates the true cost of living}
Consdier the first order expansion of $C(u_0l P_t)$ where we evaluate it where $p_T = p_0$. 
\[
C(u_0, P_t)\big|_{P_t = P_0} \approx C(u_0, P_0) + \sum_i \left. \frac{\partial C(u_0, P_t)}{\partial P_{i,t}} \right|_{P_t = P_0} (P_{i,t} - P_{i,0})
\]
Using Shephard's Lemma, we know that:
\[
\frac{\partial C(u_0, P_0)}{\partial P_{i, 0}} = Q_{i,0}(u_0,P_0)
\]
which implies that 
\[
C(u_0, P_t) \approx + \sum_i Q_{i,0} (P_{i,t} - P_{i,0}) = \sum_i P_{i,t} Q_{i,0}
\]
which implies that the Laspreyes index approximates the true cost of living. 
\subsection{Chained Index}
Over a long timer period, the assumptions that $Q_t \approx Q_0$ or $P_t \approx P_0$ are no longer tenable. This can be the idea where more goods are added to the basket at a time, and we can consider the Laspreyes index, but note that:
\[
\frac{P_{2024}}{P_{1970}} = \frac{P_{1971}}{P_{1970}} \frac{P_{1972}}{P_{1971}} \cdots \frac{P_{2023}}{P_{2022}} \frac{P_{2024}}{P_{2023}}
\]
or similarly:
\[
\frac{P_{2024}}{P_{1970}} = \frac{\sum p_{1971}q_{1970}}{\sum p_{1970}q_{1970}} \frac{\sum p_{1972} q_{1971}}{\sum p_{1971} q_{1971}}
\]
\subsection{Unemployment}
People who are over the age of 16 are only considered. People are classified as unemployed if they do not have a job, acgtively looked for work, in the prior 4 weeks, and those currently available for work. 
\todo{Make this nicer}

% % --- EXAMPLE CONTENT ---
% \section{Microeconomics Basics}

% \begin{definition}[Utility Function]
% A utility function is a representation of preferences over a set of goods and services.
% \end{definition}

% \begin{equation}
% U(x, y) = x^\alpha y^{1-\alpha}
% \end{equation}

% \begin{note}
% Cobb-Douglas functions assume constant elasticity of substitution.
% \end{note}

% \subsection{Budget Constraints}
% The budget constraint represents all combinations of goods that a consumer can afford.

% \begin{equation}
% p_x x + p_y y = I
% \end{equation}

% \subsection{Graph Example}
% \begin{center}
% \begin{tikzpicture}[scale=1]
% \draw[->] (0,0) -- (5,0) node[right] {$x$};
% \draw[->] (0,0) -- (0,5) node[above] {$y$};

% \draw[thick] (0,4) -- (4,0);
% \node at (2.5,2.2) {Budget Line};
% \end{tikzpicture}
% \end{center}

% \subsection{Market Equilibrium}
% \begin{definition}[Equilibrium]
% A state in which market supply equals market demand.
% \end{definition}

% \begin{theorem}
% Under standard assumptions, a competitive market reaches equilibrium where $D(p) = S(p)$.
% \end{theorem}

% \newpage
% \section{Macroeconomics Overview}
% % Add more sections as needed...

\end{document}
