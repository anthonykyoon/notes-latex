\documentclass[11pt]{article}
\usepackage{graphicx}
\usepackage[utf8]{inputenc}
\usepackage{geometry}
\usepackage{titlesec}
\usepackage{enumitem}
\usepackage{hyperref} 
\usepackage{amsmath}
\usepackage{ gensymb }
\usepackage{ amssymb }
\usepackage{float}
\usepackage{amsthm}
\usepackage{float}

\newcommand{\R}{\mathbb{R}}
\newcommand{\N}{\mathbb{N}}
\newcommand{\Q}{\mathbb{Q}}
\newcommand{\Z}{\mathbff{Z}}
\newcommand{\st}{\text{s.t}}

\title{PSET 5 and 6}
\author{Anthony Yoon}
\date{3/1/2025}
\begin{document}
\maketitle
\section{}
Set of players: $\{ \text{Frank Underwood}, \text{Raymond Tusk}\} = \{p1, p2\}$.\\ 
Set of actions: $\{\text{Compromise}, \text{Not Compromise}\} = \{ C, NC\}$ \\
We can see that:
\[
u_1(C, C) = 3 \quad u_2(C,C) = 3 
\]
\[
u_1(C, NC) = 1 \quad u_2(C, NC) = 2
\]
\[
u_1(NC, C) = 2 \quad u_2(NC, C) = 1
\]
\[
u_2(NC, NC) = 0 \quad u_2(NC, NC) = 0
\]
The table is as follows:
\begin{table}[h]
    \centering 
    \begin{tabular}{c|c|c}
        & C & NC\\
        \hline
        C & $(3,3)$ & $(1,2)$\\
        NC & $(2,1)$ & $(0,0)$ 
    \end{tabular}
\end{table}
We can see that the Nash Equilbrium is $(3,3)$
\section{}
\section{}
\section{}
\section{}
\section{}
\section{}
\section{}
\end{document}