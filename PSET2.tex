\documentclass[11pt]{article}
\usepackage{graphicx}
\usepackage[utf8]{inputenc}
\usepackage{geometry}
\geometry{a4paper, margin=0.75in}
\usepackage{titlesec}
\usepackage{enumitem}
\usepackage{hyperref} 
\usepackage{amsmath}
\usepackage{ gensymb }
\usepackage{ amssymb }
\usepackage{float}
\usepackage{amsthm}
\usepackage{float}

\newcommand{\st}{\text{s.t}}
\newcommand{\R}{\mathbb{R}}
\newcommand{\N}{\mathbb{N}}

\title{ECON 20110 PSET 2}
\author{Anthony Yoon}
\date{1/24/2025}
\begin{document}
\maketitle

\section*{1}
\subsection*{a}
Let $y_1$ denote cotton threads, such that $y_1 \in Y$. Let $x_1$ denote raw cottoon such that $x_1 \in C$ and $x_2 \in L$. Therefore we see that the production possibility set that describes Firm 1's technology is $F_1 \subseteq X \times Y$ where $F_1 = \{ (x_1, x_2, y_1) \in \R^2_+ \hspace{2pt} \}| \hspace{2pt} y_1 \leq x^{\frac{1}{2}}_1 x^{\frac{1}{2}}_2 \}$. 
\subsection*{b}
Let $y_1$ and $Y$ be denoted the same as before. Let $y_2 \in P$ where $y_2$ is the number of pillow cases and $P$ is the set of all pillow cases. Thus, the production possiblity set denoted by $F_2$ and is $F_2 \subseteq Y \times P$, where $F_2 = \{ (y_1, y_2) \in \R^2_+ \hspace{2pt} | \hspace{2pt} y_2 \leq y_1^{\frac{1}{2}}\}$. 
\subsection*{c}
Logically, this would be the union of these sets. Since no goods are being produced, we do not have to be concerend about any new goods and be soley be concerned about the inputs. Let $F_3$ denote Firm 3's production set. We can see that $F_3 = F_2 \cup F_1$ and $F_3 = \{(x_1, x_2, y_1, y_2) \in \R^4_+ \hspace{2pt} | \hspace{2pt} y_1 \leq x_1^{\frac{1}{2}} x_2^{\frac{1}{2}}, y_2 \leq y_1^{\frac{1}{2}}\}$.
\section*{2}
\subsection*{a}

\section*{3}
\section*{4}
\section*{5}


\end{document}